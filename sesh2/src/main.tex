\documentclass[a4paper]{article}

\usepackage[T1]{fontenc}
\usepackage[utf8]{inputenc}

\usepackage{mathptmx}

\usepackage[a4paper, total={6in, 8in}]{geometry}
\usepackage{subcaption}
\usepackage[shortlabels]{enumitem}
\usepackage{amssymb}
\usepackage{amsthm}
\usepackage{mathtools}
\usepackage{bbm}
\usepackage{graphicx}
\usepackage{float}
\usepackage[colorlinks=true,naturalnames=true,plainpages=false,pdfpagelabels=true]{hyperref}
\usepackage[parfill]{parskip}
\usepackage[backend=biber, sorting=none]{biblatex}
\addbibresource{uni.bib}
\pagestyle{myheadings}
\markright{Popovic, Vogel\hfill Unbiased Fitting \hfill}

\title{Theoretical Physics Lab-Course 2021S\\ University of Vienna \vspace{1.25cm}\\
Quantum Entanglement at High Energies}
\author{Milutin Popovic \\ Tim Vogel \vspace{1cm}\\ Supervisor: Prof. Dr. Beatrix C.
Hiesmayr}
\date{April 18, 2021}

\begin{document}
\maketitle
\tableofcontents

\section{Bell Theorem and Bell Inequality}
The Enstein-Podolsky-Rosen paradox argues that, because of hidden variables
not considered, quantum mechanics is incomplete. J.S. Bell in his paper \cite{bell},
published in 1964 discovered what is known as the Bell's Theorem. The theorem
states in short that "In certain experiments all local realistic theories are
incompatible with quantum mechanics" \cite{bell}. This is achieved by
establishing a so called Bell inequality which satysiefs the local realistic
theories but violates quantum mechanics.

To demonstrate the results of Bell, we consider a Wiegner-type Bell inequality
for spin-$\frac{1}{2}$ particles with a source that generates a Bell state
\begin{align}\label{eq:wtb}
    P(\Uparrow \vec{a}, \Uparrow \vec{b}) \leq
    P(\Uparrow \vec{a}, \Uparrow \vec{c}) + P(\Uparrow \vec{c}, \Uparrow \vec{b})
\end{align}

here we denote the joint probability, of say Alice and Bob to find their
particles in the spin-up state, with respect to their orientation $\vec{a},
\vec{b}$, $P(\Uparrow \vec{a}, \Uparrow \vec{b})$.For the source we consider
an antisymmetric Bell state $|\psi ^-\rangle$ in terms of $\vec{a}$.
\begin{align}
    |\psi ^-\rangle = \frac{1}{\sqrt{2}}(|\Uparrow \vec{a}; \Downarrow
    \vec{a}\rangle - |\Downarrow \vec{a}; \Uparrow \vec{a}\rangle).
\end{align}
Quantum Mechanics tells us how to calculate the probability
\begin{align}\label{eq:prob}
    P^{QM}(\Uparrow \vec{a}, \Uparrow \vec{b}) = \left\|\langle\Uparrow \vec{a}; \Uparrow
    \vec{b}|\psi ^-\rangle \right\|^2.
\end{align}

Furthermore quantum mechanics tells us how we can represent the ket
$|\Uparrow \vec{b}\rangle$ in terms $|\Uparrow \vec{a}\rangle$ and the other
way around.
\begin{align}
    |\Uparrow \vec{b}\rangle &= \cos(\frac{\phi_{ab}}{2})|\Uparrow
    \vec{a}\rangle + e^{i\theta_{ab}} \sin(\frac{\phi_{ab}}{2})|\Downarrow
    \vec{a}\rangle\\
    |\Uparrow \vec{a}\rangle &= \sin(\frac{\phi_{ab}}{2})|\Uparrow
    \vec{b}\rangle - e^{i\theta_{ab}} \cos(\frac{\phi_{ab}}{2})|\Downarrow
    \vec{b}\rangle
\end{align}

where $\theta_{ab}$ is an unphysical phase and can be set to zero.

With this information we can derive the following results for the
probability in Equation \ref{eq:prob} (Task 1)
\begin{align}
    P^{QM}(\Uparrow \vec{a}, \Uparrow \vec{b}) &= \left\|\langle\Uparrow \vec{a}; \Uparrow
    \vec{b}|\psi ^-\rangle \right\|^2 \\
    &=\frac{1}{2} \left\|\langle\Uparrow \vec{a}; \Uparrow \vec{b}|\Uparrow
    \vec{a}; \Downarrow \vec{a}\rangle - \langle \Uparrow \vec{a}; \Uparrow
    \vec{b}| \Downarrow \vec{a}; \Uparrow \vec{a}\rangle\right\|^2 \\
    &= \frac{1}{2}\left\|(\langle\Uparrow \vec{a}|\Uparrow\vec{a}\rangle)
    (\langle\Uparrow \vec{b}| \Downarrow \vec{a}\rangle) - (\underbrace{\langle\Uparrow
    \vec{a}| \Downarrow \vec{a}\rangle}_{\bot})(\langle\Uparrow \vec{b}|
    \Uparrow\vec{a}\rangle)\right\|^2 \\
    &=\frac{1}{2}\left\|\langle\Uparrow\vec{b}|\Downarrow\vec{a}\rangle\right\|^2
    =\frac{1}{2}\sin^2(\frac{\phi_{ab}}{2})
\end{align}

For the remaining probabilities in $|\Uparrow \vec{a}\Downarrow \vec{b}\rangle$,$|\Downarrow
\vec{a}\Uparrow\vec{b}\rangle$ and $|\Downarrow\vec{a}\Downarrow\vec{b} \rangle$
we apply the same procedure to get
\begin{align}
    P^{QM}(\Uparrow \vec{a}, \Downarrow \vec{b}) &= \left\|\langle\Uparrow
    \vec{a}; \Downarrow \vec{b}|\psi^- \rangle\right\|^2 \\
    &=\frac{1}{2}\left\|\langle\Downarrow \vec{b}|\Downarrow
    \vec{a}\rangle\right\|^2 = \frac{1}{2}\cos^2(\frac{\phi_{ab}}{2})\\
    \\
    P^{QM}(\Downarrow \vec{a}, \Downarrow \vec{b}) &= \left\|\langle\Downarrow
    \vec{a}; \Downarrow \vec{b}|\psi^- \rangle\right\|^2 \\
    &=\frac{1}{2}\left\|\langle\Downarrow \vec{b}|\Uparrow
    \vec{a}\rangle\right\|^2 = \frac{1}{2}\sin^2(\frac{\phi_{ab}}{2})
\end{align}

Thus for the Bell inequality we get

\begin{align}
    \frac{1}{2}sin^2(\frac{\phi_{ab}}{2}) \leq
    \frac{1}{2}sin^2(\frac{\phi_{ac}}{2})+
    \frac{1}{2}sin^2(\frac{\phi_{cb}}{2})
\end{align}
We can take $\vec{a}$, $\vec{b}$ to be in the same plane in $\mathbb{R}^3$ and
$\vec{c}$ ($\in \mathbb{R}^2$) to be the bisector of these two. This would imply
$\phi_{ac}=\phi_{cb}= \phi$ and $\phi_{ab}=2\phi$, which gives us.
\begin{align}
    \sin^2(\phi) &\leq 2\sin^2(\frac{\phi}{2})\\
    \Leftrightarrow \cos(\frac{\phi}{2}) &\leq \frac{1}{2}
\end{align}
This is not always the case and is thereby an obvious contradiction.

Now we will derive the original inequality of John Stuart Bell\cite{bell}.
For this we consider the expectation values. The expectation value $E$ is expressed
in terms of probabilities of all possible outcomes.
\begin{align}
    E(\vec{a}, \vec{b}) &= P(\Uparrow \vec{a}, \Uparrow \vec{b}) + P(\Downarrow
    \vec{a}, \Downarrow \vec{b}) -P(\Uparrow \vec{a}, \Downarrow \vec{b})
    -P(\Downarrow \vec{a}, \Uparrow \vec{b})\\
    E(\vec{a}, \vec{c}) &= P(\Uparrow \vec{a}, \Uparrow \vec{c}) + P(\Downarrow
    \vec{a}, \Downarrow \vec{c}) -P(\Uparrow \vec{a}, \Downarrow \vec{c})
    -P(\Downarrow \vec{a}, \Uparrow \vec{c})\\
    E(\vec{c}, \vec{b}) &= P(\Uparrow \vec{c}, \Uparrow \vec{b}) + P(\Downarrow
    \vec{c}, \Downarrow \vec{b}) -P(\Uparrow \vec{c}, \Downarrow \vec{b})
    -P(\Downarrow \vec{c}, \Uparrow \vec{b})
\end{align}
The sum of spin-up/spin-down and spin-down/spin-up probabilities needs to be
one, with this we get.
\begin{align}
    P(\Uparrow \vec{a}, \Uparrow \vec{b}) = E(\vec{a}, \vec{b}) - P(\Downarrow
    \vec{a}, \Downarrow \vec{b}) + 1\\
    P(\Uparrow \vec{a}, \Uparrow \vec{c}) = E(\vec{a}, \vec{c}) - P(\Downarrow
    \vec{a}, \Downarrow \vec{c}) + 1\\
    P(\Uparrow \vec{c}, \Uparrow \vec{b}) = E(\vec{c}, \vec{b}) - P(\Downarrow
    \vec{c}, \Downarrow \vec{b}) + 1
\end{align}
Plugging this into the inequality in equation \ref{eq:wtb} we get
\begin{align}
    E(\vec{a}, \vec{b}) - E(\vec{a}, \vec{c}) &\leq E(\vec{c}, \vec{b}) + 1\\
    & + P(\Downarrow \vec{a}, \Downarrow \vec{b}) -P(\Downarrow \vec{a},
    \Downarrow \vec{c}) -P(\Downarrow \vec{c}, \Downarrow \vec{b})
    \label{eq:rightside}
\end{align}
The term in equation \ref{eq:rightside} satisfies the bell inequality if the
term on the left is strictly positive. Thus we get the original Bell inequality
\begin{align}
    |E(\vec{a}, \vec{b}) - E(\vec{a}, \vec{c})| \leq E(\vec{c}, \vec{b}) + 1
\end{align}
\section{Quantum System of K-mesons}
Kaons were discovered in 1947 by Rochester and Butler, where cosmic ray
particles hit a lead plate and produced a neutral particle. The neutral
particle is named kaon, it was noticed through its decay in two charged
pions. $K^0$, the kaon, is the first strange particle.
The neutral kaons are states of quarks and anti-quarks, the ket
$|K^0\rangle = |d\bar{s}\rangle$ denotes the neutral K-meson, and its
antiparticle is $|\bar{K}^0\rangle = |\bar{d}s\rangle$. The state of the kaon
is entangled in the sense that the particle $K^0$ can turn into an antiparticle
$\bar{K}^0$ before decay through strangeness oscillation, a particle-antiparticle oscillation.


There are two states, the long lived kaon $K_L$ and the short lived kaon $K_S$
which diagonalize the Hamiltonian,
the difference of their decay rates is about a factor of $600$.
The thing is $K_L$ should decay into three pions, while $K_S$ should decay
into two pions. But in 1964 Cronin and Fitch found out that the long lived
kaon can also decay into two pions, which directly implies a $CP$ symmetry violation.
$CP$ symmetry (charge conjugation parity symmetry) in particle physics states
that physics is the same if instead of looking at an particle we looked at its
antiparticle.

The short lived kaon $K_S$ and the long lived kaon $K_L$ can be expressed in
terms of $|K^0\rangle$ and $|\bar{K^0}\rangle$ as follows,

\begin{align}
    |K_S\rangle &=  \frac{1}{N}(p|K^0\rangle - q|\bar{K}^0\rangle) \\
    |K_L\rangle &=  \frac{1}{N}(p|K^0\rangle + q|\bar{K}^0\rangle)
\end{align}

Where $p = 1+\varepsilon$, $q=1-\varepsilon$ and $N^2 = |p|^2 + |q|^2$,
$\varepsilon$ is called the $CP$ violating parameter and can be measured, with
a magnitude of $|\varepsilon| \approx 10^{-3}$ \cite{Bertlmann} . Note
that these two states are NOT orthogonal due to the $CP$ violation.


Furthermore the connection to the $CP$ basis is,
\begin{align}
    |K_1\rangle &=  \frac{1}{\sqrt{2}}(|K^0\rangle - e^{i\alpha}|\bar{K}^0\rangle) \\
    |K_2\rangle &=  \frac{1}{\sqrt{2}}(|K^0\rangle + e^{i\alpha}|\bar{K}^0\rangle) \\
\end{align}
where $\alpha$ is an unphysical phase and is conventionally set to zero.


\subsection{CP-Symmetry violation}
The violation in symmetry can be seen considering the Bell inequality
\begin{align}\label{eq:leqk}
    P(K_S, \bar{K}^0) \leq P(K_S, K_1) + P(K_1, \bar{K}^0)
\end{align}
where $P(K_S, \bar{K}^0)$ is the probability of finding $K_S$ on the right and
$\bar{K}^0$ on the left at the time of measurement at $t=0$, $K_1$ denotes a
completely unphysical state and cannot be measured. Quantum mechanics gives us
the tools to calculate these probabilities, the calculation gives
\begin{align}
    |p|&\leq |q|\\
    \Leftrightarrow \delta &:= \frac{|p|^2 - |q|^2}{|p|^2+|q|^2} \leq 0
\end{align}

Changing the choice of $\bar{K}^0$ and $K^0$ in equation \ref{eq:leqk} we
calculate the probabilities:
\begin{align}
    P^{QM}(K_S, K^0)&=\left\|\langle K_S, K^0|\psi ^-\rangle\right\|
    ^2=\frac{1}{2}\frac{|q|^2}{2N^2} \\
    P^{QM}(K_S, K_1)&= \left\|\langle K_S, K_1|\psi ^-\rangle\right\|^2=
    \frac{1}{4N^2}\left|q-pe^{i\alpha}\right|\\
    P^{QM}(K_1, K^0)&=\left\|\langle K_1, K^0|\psi ^-\rangle\right\| ^2=
    \frac{1}{4}|e^{i\alpha}|^2
\end{align}
Inserting this into the inequality in equation \ref{eq:leqk} and applying some
basic algebra we get
\begin{align}
    |q| &\leq |p|\\
    \Leftrightarrow \delta &= \frac{|p|^2 - |q|^2}{|p|^2+|q|^2} \geq 0
\end{align}
implying a strict equality
\begin{align}
        \delta = 0.
\end{align}
This contradicts the experimental value of $\delta$.
\begin{align}
    \delta_{exp} = (3.27\pm 0.12)\cdot 10^{-1}.
\end{align}

We can say that the CP-violation is directly related to the violation of the
Bell Inequality and entanglement.
If CP-symmetry were true the long lived kaon would decay via two paths in equal
amounts, which is not the case. Also
according to the Big-Bang-Theory the amount
of matter and antimatter initially created is equal, but according to
experimental results CP-asymmetry means that there is an imbalance in
matter and that physics differs for particles and antiparticles.




\printbibliography
\end{document}
