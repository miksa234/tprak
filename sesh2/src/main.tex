\documentclass[a4paper]{article}

\usepackage[T1]{fontenc}
\usepackage[utf8]{inputenc}

\usepackage{mathptmx}

\usepackage[a4paper, total={6in, 8in}]{geometry}
\usepackage{subcaption}
\usepackage[shortlabels]{enumitem}
\usepackage{amssymb}
\usepackage{amsthm}
\usepackage{mathtools}
\usepackage{braket}
\usepackage{bbm}
\usepackage{graphicx}
\usepackage{float}
\usepackage[colorlinks=true,naturalnames=true,plainpages=false,pdfpagelabels=true]{hyperref}
\usepackage[parfill]{parskip}
\usepackage[backend=biber, sorting=none]{biblatex}
\addbibresource{uni.bib}
\pagestyle{myheadings}
\markright{Popovic, Vogel\hfill Unbiased Fitting \hfill}

\title{University of Vienna\\ Faculty of Physics\\ \vspace{1.25cm} Theoretical Physics Lab-Course 2021S\\
Quantum Entanglement at High Energies}
\author{Milutin Popovic \& Tim Vogel \vspace{1cm}\\ Supervisor: Prof. Dr. Beatrix C.
Hiesmayr}
\date{May the 9th, 2021}

\begin{document}
\maketitle
\noindent\rule[0.5ex]{\linewidth}{1pt}
\begin{abstract}
In this article we go through the lecture notes of Prof. Dr. Beatrix C.
Hiesmayr and the exercises given to us in the Theoretical Physics Lab-Course
in the summer semester of  2021. Precisely  we look at entangled systems at high
energies, we learn about the Bells inequality and it's violation within the world
of quantum mechanics.
Furthermore we learn about the density matrix approach in quantum mechanics,
which allows us to look at the unstable particles with different decay rates
e.g. the K-meson.
\end{abstract}
\noindent\rule[0.5ex]{\linewidth}{1pt}

\tableofcontents

\section{Bell Theorem and Bell Inequality}
The Enstein-Podolsky-Rosen paradox argues that, because of hidden variables
not considered, quantum mechanics is incomplete. J.S. Bell in his paper \cite{bell},
published in 1964 discovered what is known as the Bell's Theorem. The theorem
states in short that "In certain experiments all local realistic theories are
incompatible with quantum mechanics" \cite{bell}. This is achieved by
establishing a so called Bell inequality which satisfies the local realistic
theories but violates quantum mechanics.

To demonstrate the results of Bell, we consider a Wiegner-type Bell inequality
for spin-$\frac{1}{2}$ particles with a source that generates a Bell state
\begin{align}\label{eq:wtb}
    P(\Uparrow \vec{a}, \Uparrow \vec{b}) \leq
    P(\Uparrow \vec{a}, \Uparrow \vec{c}) + P(\Uparrow \vec{c}, \Uparrow \vec{b})
\end{align}

here we denote the joint probability, of say Alice and Bob to find their
particles in the spin-up state, with respect to their orientation $\vec{a},
\vec{b}$, $P(\Uparrow \vec{a}, \Uparrow \vec{b})$. For the source we consider
an antisymmetric Bell state $|\psi ^-\rangle$ in terms of $\vec{a}$.
\begin{align}
    |\psi ^-\rangle = \frac{1}{\sqrt{2}}(|\Uparrow \vec{a}; \Downarrow
    \vec{a}\rangle - |\Downarrow \vec{a}; \Uparrow \vec{a}\rangle).
\end{align}
Quantum Mechanics tells us how to calculate the probability
\begin{align}\label{eq:prob}
    P^{QM}(\Uparrow \vec{a}, \Uparrow \vec{b}) = \left\|\langle\Uparrow \vec{a}; \Uparrow
    \vec{b}|\psi ^-\rangle \right\|^2.
\end{align}

Furthermore quantum mechanics tells us how we can represent the ket
$|\Uparrow \vec{b}\rangle$ in terms $|\Uparrow \vec{a}\rangle$ and the other
way around.
\begin{align}
    |\Uparrow \vec{b}\rangle &= \cos(\frac{\phi_{ab}}{2})|\Uparrow
    \vec{a}\rangle + e^{i\theta_{ab}} \sin(\frac{\phi_{ab}}{2})|\Downarrow
    \vec{a}\rangle\\
    |\Uparrow \vec{a}\rangle &= \sin(\frac{\phi_{ab}}{2})|\Uparrow
    \vec{b}\rangle - e^{i\theta_{ab}} \cos(\frac{\phi_{ab}}{2})|\Downarrow
    \vec{b}\rangle
\end{align}

where $\theta_{ab}$ is an unphysical phase in this case and is set to zero.

With this information we can derive the following results for the
probability in Equation \ref{eq:prob} (Task 1)
\begin{align}
    P^{QM}(\Uparrow \vec{a}, \Uparrow \vec{b}) &= \left\|\langle\Uparrow \vec{a}; \Uparrow
    \vec{b}|\psi ^-\rangle \right\|^2 \\
    &=\frac{1}{2} \left\|\langle\Uparrow \vec{a}; \Uparrow \vec{b}|\Uparrow
    \vec{a}; \Downarrow \vec{a}\rangle - \langle \Uparrow \vec{a}; \Uparrow
    \vec{b}| \Downarrow \vec{a}; \Uparrow \vec{a}\rangle\right\|^2 \\
    &= \frac{1}{2}\left\|(\langle\Uparrow \vec{a}|\Uparrow\vec{a}\rangle)
    (\langle\Uparrow \vec{b}| \Downarrow \vec{a}\rangle) - (\underbrace{\langle\Uparrow
    \vec{a}| \Downarrow \vec{a}\rangle}_{\bot})(\langle\Uparrow \vec{b}|
    \Uparrow\vec{a}\rangle)\right\|^2 \\
    &=\frac{1}{2}\left\|\langle\Uparrow\vec{b}|\Downarrow\vec{a}\rangle\right\|^2
    =\frac{1}{2}\sin^2(\frac{\phi_{ab}}{2})
\end{align}

For the remaining probabilities in $|\Uparrow \vec{a}\Downarrow \vec{b}\rangle$,$|\Downarrow
\vec{a}\Uparrow\vec{b}\rangle$ and $|\Downarrow\vec{a}\Downarrow\vec{b} \rangle$
we apply the same procedure to get
\begin{align}
    P^{QM}(\Uparrow \vec{a}, \Downarrow \vec{b}) &= \left\|\langle\Uparrow
    \vec{a}; \Downarrow \vec{b}|\psi^- \rangle\right\|^2 \\
    &=\frac{1}{2}\left\|\langle\Downarrow \vec{b}|\Downarrow
    \vec{a}\rangle\right\|^2 = \frac{1}{2}\cos^2(\frac{\phi_{ab}}{2})\\
    \\
    P^{QM}(\Downarrow \vec{a}, \Downarrow \vec{b}) &= \left\|\langle\Downarrow
    \vec{a}; \Downarrow \vec{b}|\psi^- \rangle\right\|^2 \\
    &=\frac{1}{2}\left\|\langle\Downarrow \vec{b}|\Uparrow
    \vec{a}\rangle\right\|^2 = \frac{1}{2}\sin^2(\frac{\phi_{ab}}{2})
\end{align}

Thus for the Bell inequality we get

\begin{align}
    \frac{1}{2}sin^2(\frac{\phi_{ab}}{2}) \leq
    \frac{1}{2}sin^2(\frac{\phi_{ac}}{2})+
    \frac{1}{2}sin^2(\frac{\phi_{cb}}{2})
\end{align}
We can take $\vec{a}$, $\vec{b}$ to be in the same plane in $\mathbb{R}^3$ and
$\vec{c}$ ($\in \mathbb{R}^2$) to be the bisector of these two. This would imply
$\phi_{ac}=\phi_{cb}= \phi$ and $\phi_{ab}=2\phi$, which gives us.
\begin{align}
    \sin^2(\phi) &\leq 2\sin^2(\frac{\phi}{2})\\
    \Leftrightarrow \cos(\frac{\phi}{2}) &\leq \frac{1}{2}
\end{align}
This is not always the case and is thereby an obvious contradiction.

Now we will derive the original inequality of John Stuart Bell\cite{bell}.
For this we consider the expectation values. The expectation value $E$ is expressed
in terms of probabilities of all possible outcomes.
\begin{align}
    E(\vec{a}, \vec{b}) &= P(\Uparrow \vec{a}, \Uparrow \vec{b}) + P(\Downarrow
    \vec{a}, \Downarrow \vec{b}) -P(\Uparrow \vec{a}, \Downarrow \vec{b})
    -P(\Downarrow \vec{a}, \Uparrow \vec{b})\\
    E(\vec{a}, \vec{c}) &= P(\Uparrow \vec{a}, \Uparrow \vec{c}) + P(\Downarrow
    \vec{a}, \Downarrow \vec{c}) -P(\Uparrow \vec{a}, \Downarrow \vec{c})
    -P(\Downarrow \vec{a}, \Uparrow \vec{c})\\
    E(\vec{c}, \vec{b}) &= P(\Uparrow \vec{c}, \Uparrow \vec{b}) + P(\Downarrow
    \vec{c}, \Downarrow \vec{b}) -P(\Uparrow \vec{c}, \Downarrow \vec{b})
    -P(\Downarrow \vec{c}, \Uparrow \vec{b})
\end{align}
The sum of spin-up/spin-down and spin-down/spin-up probabilities needs to be
one, with this we get.
\begin{align}
    P(\Uparrow \vec{a}, \Uparrow \vec{b}) = E(\vec{a}, \vec{b}) - P(\Downarrow
    \vec{a}, \Downarrow \vec{b}) + 1\\
    P(\Uparrow \vec{a}, \Uparrow \vec{c}) = E(\vec{a}, \vec{c}) - P(\Downarrow
    \vec{a}, \Downarrow \vec{c}) + 1\\
    P(\Uparrow \vec{c}, \Uparrow \vec{b}) = E(\vec{c}, \vec{b}) - P(\Downarrow
    \vec{c}, \Downarrow \vec{b}) + 1
\end{align}
Plugging this into the inequality in equation \ref{eq:wtb} we get
\begin{align}
    E(\vec{a}, \vec{b}) - E(\vec{a}, \vec{c}) &\leq E(\vec{c}, \vec{b}) + 1\\
    & + P(\Downarrow \vec{a}, \Downarrow \vec{b}) -P(\Downarrow \vec{a},
    \Downarrow \vec{c}) -P(\Downarrow \vec{c}, \Downarrow \vec{b})
    \label{eq:rightside}
\end{align}
The term in equation \ref{eq:rightside} satisfies the bell inequality if the
term on the left is strictly positive. Thus we get the original Bell inequality
\begin{align}
    |E(\vec{a}, \vec{b}) - E(\vec{a}, \vec{c})| \leq E(\vec{c}, \vec{b}) + 1
\end{align}
\section{Quantum System of K-mesons}
Kaons were discovered in 1947 by Rochester and Butler, where cosmic ray
particles hit a lead plate and produced a neutral particle. The neutral
particle is named kaon, it was noticed through its decay in two charged
pions. $K^0$, the kaon, is the first strange particle.
The neutral kaons are states of quarks and anti-quarks, the ket
$|K^0\rangle = |d\bar{s}\rangle$ denotes the neutral K-meson, and its
antiparticle is $|\bar{K}^0\rangle = |\bar{d}s\rangle$. The state of the kaon
is entangled in the sense that the particle $K^0$ can turn into an antiparticle
$\bar{K}^0$ before decay through strangeness oscillation, a particle-antiparticle oscillation.


There are two states, the long lived kaon $K_L$ and the short lived kaon $K_S$
which diagonalize the Hamiltonian,
the difference of their decay rates is about a factor of $600$.
The thing is $K_L$ should decay into three pions, while $K_S$ should decay
into two pions. But in 1964 Cronin and Fitch found out that the long lived
kaon can also decay into two pions, which directly implies a $CP$ symmetry violation.
$CP$ symmetry (charge conjugation parity symmetry) in particle physics states
that physics is the same if instead of looking at an particle we looked at its
antiparticle.

The short lived kaon $K_S$ and the long lived kaon $K_L$ can be expressed in
terms of $|K^0\rangle$ and $|\bar{K^0}\rangle$ as follows,

\begin{align}
    |K_S\rangle &=  \frac{1}{N}(p|K^0\rangle - q|\bar{K}^0\rangle) \\
    |K_L\rangle &=  \frac{1}{N}(p|K^0\rangle + q|\bar{K}^0\rangle)
\end{align}
where $p = 1+\varepsilon$, $q=1-\varepsilon$ and $N^2 = |p|^2 + |q|^2$,
$\varepsilon$ is called the $CP$ violating parameter and can be measured, with
a magnitude of $|\varepsilon| \approx 10^{-3}$ \cite{Bertlmann} . Note
that these two states are NOT orthogonal due to the $CP$ violation.


Furthermore the connection to the $CP$ basis is,
\begin{align}
    |K_1\rangle &=  \frac{1}{\sqrt{2}}(|K^0\rangle - e^{i\alpha}|\bar{K}^0\rangle) \\
    |K_2\rangle &=  \frac{1}{\sqrt{2}}(|K^0\rangle + e^{i\alpha}|\bar{K}^0\rangle) \\
\end{align}
here $\alpha$ is an unphysical phase and is conventionally set to zero.


\subsection{CP-Symmetry violation}
The violation in symmetry can be seen considering the Bell inequality
\begin{align}\label{eq:leqk}
    P(K_S, \bar{K}^0) \leq P(K_S, K_1) + P(K_1, \bar{K}^0)
\end{align}
where $P(K_S, \bar{K}^0)$ is the probability of finding $K_S$ on the right and
$\bar{K}^0$ on the left at the time of measurement at $t=0$, $K_1$ denotes a
completely unphysical state and cannot be measured. Quantum mechanics gives us
the tools to calculate these probabilities, the calculation gives
\begin{align}
    |p|&\leq |q|\\
    \Leftrightarrow \delta &:= \frac{|p|^2 - |q|^2}{|p|^2+|q|^2} \leq 0
\end{align}

Changing the choice of $\bar{K}^0$ and $K^0$ in equation \ref{eq:leqk} we
calculate the probabilities again:
\begin{align}
    P^{QM}(K_S, K^0)&=\left\|\langle K_S, K^0|\psi ^-\rangle\right\|
    ^2=\frac{|q|^2}{4N^2} \\
    P^{QM}(K_S, K_1)&= \left\|\langle K_S, K_1|\psi ^-\rangle\right\|^2=
    \frac{1}{4N^2}\left|q-pe^{i\alpha}\right|\\
    P^{QM}(K_1, K^0)&=\left\|\langle K_1, K^0|\psi ^-\rangle\right\| ^2=
    \frac{1}{4}|e^{i\alpha}|^2
\end{align}
Inserting this into the inequality in equation \ref{eq:leqk} and applying some
basic algebra we get
\begin{align}
    |q| &\leq |p|\\
    \Leftrightarrow \delta &= \frac{|p|^2 - |q|^2}{|p|^2+|q|^2} \geq 0
\end{align}
implying a strict equality
\begin{align}
        \delta = 0.
\end{align}
This contradicts the experimental value of $\delta$ (citation from slides).
\begin{align}
    \delta_{exp} = (3.27\pm 0.12)\cdot 10^{-1}.
\end{align}

We can say that the CP-violation is directly related to the violation of the
Bell Inequality and entanglement.
If CP-symmetry were true the long lived kaon would decay via two paths in equal
amounts, which is not the case. Also
according to the Big-Bang-Theory the amount
of matter and antimatter initially created is equal, but according to
experimental results CP-asymmetry means that there is an imbalance in
matter and that physics differs for particles and antiparticles.

\subsection{Efficient description of decaying quantum systems}
The neutral kaon system, is usually described by an effective Schrödinger-equation,
which is given by the Lioville von Neumann form as:
\begin{equation}
    \frac{d}{dt}\rho=-iH_{eff}\rho+i\rho H^\dagger_{eff}
\end{equation}
where $\rho$ is a 2x2 density matrix and $H_{eff}$ non hermitian. The Hamiltonian can
be decomposed via the Wigner-Weisskopf approximation into: $H_{eff}=M-\frac{i}{2}\Gamma$,
with the 2x2 mass matrix $H$ and the 2x2 decay matrix $\Gamma$ both being positive and hermitian.
What we are now concerned with, is the question, whether $H^\dagger_{eff}$
can also be decomposed and the implications the result gives.
Starting from the decomposed $H_{eff}$, applying the dagger, we get:
\begin{equation}
    H^\dagger_{eff}=M^T+\frac{i}{2}\Gamma^T
\end{equation}
which leads to:
\begin{equation}
    H^\dagger_{eff}=\left( \begin{array}{cc}
        M_0+\frac{i}{2}\Gamma_0 & (M_{12})^*+\frac{i}{2}(\Gamma_{12})^* \\
        (M_{12})+\frac{i}{2}(\Gamma_{12})^* & M_0+\frac{i}{2}\Gamma_0
    \end{array}\right)
\end{equation}
This final matrix can now be brought into the form:
\begin{equation}
    \begin{pmatrix}
        A^* & B^*r \\
        \frac{B^*}{r} & A^*
    \end{pmatrix}
\end{equation}
with $A,B and r$ being complex numbers.
\newline
We now compute the Eigenvalues of this matrix, giving:
\begin{equation}
    (A^*-\lambda)^2-(B^*)^2 \rightarrow \lambda=A^*\pm B^*
\end{equation}
With this, the Eigenvectors take the form:
\begin{equation}
    v_1=\begin{pmatrix}
        r^* \\
        1
    \end{pmatrix}
\end{equation}
and
\begin{equation}
    v_2=\begin{pmatrix}
        -r^* \\
        1
    \end{pmatrix}
\end{equation}
We now define the matrices:
\begin{equation}
    R^{-1}=
  \begin{pmatrix}
        r^* & -r^* \\
        1 & 1
    \end{pmatrix}
\end{equation}
and
\begin{equation}
    R= \frac{1}{2r}
    \begin{pmatrix}
        1 & r^* \\
        -1 & r^*
    \end{pmatrix}
\end{equation}
and with
\begin{equation}
    RH^\dagger_{eff}R^{-1}=
    \begin{pmatrix}
        A^*+B^* & 0 \\
        0 & A^*-B^*
    \end{pmatrix}
\end{equation}
we finally find values for $\ket{K_S}$ and $\ket{K_L}$ which explicitly are:
\begin{equation}
    \ket{K_S}=\frac{1}{\sqrt{1+|r|^2}}(-r^*\ket{K^0}+\ket{\bar{K^0}})
\end{equation}
and
\begin{equation}
    \ket{K_L}=\frac{1}{1+|r|^2}(r^*\ket{K^0}-\ket{\bar{K^0}}
\end{equation}
We now want to calculate the overlap of these eigenvectors. We start with the
overlap in $H_{eff}$, which is given by:
\begin{equation}
    \braket{K_S|K_L}=\frac{1-|r|^2}{1+|r|^2}
\end{equation}
We now define $\varepsilon=\frac{1-r}{1+r}\rightarrow r=\frac{1-\varepsilon}{1+\varepsilon}$, which leads to:
\begin{equation}
    \braket{K_S|K_L}=\frac{|\varepsilon+1|^2-|\varepsilon-1|^2}{|\varepsilon+1|^2+|\varepsilon-1|^2}
\end{equation}
And with only considering the real part of Epsilon, we finally obtain:
\begin{equation}
    \braket{K_S|K_L}=\frac{2Re(\varepsilon}{|\varepsilon|^2+1})
\end{equation}
We now do the same for the overlap in $H^\dagger_{eff}$, and thereby obtain the result:
\begin{equation}
    \braket{K_S|K_L}=\frac{1-|r|^2}{1+|r|^2}=\frac{2Re(\varepsilon)}{1+|\varepsilon|^2}
\end{equation}
This means, that in both cases we obtain a CP-violation.
\subsection{Charge asymmetry}
Finally, we look at the following charge asymmetry term, given by:
\begin{equation}
    \delta(t)=\frac{P(K^0,t;|K^0|)-P(\bar{K^0},t;|K^0|)}{P(K^0,t;|K^0|)+P(\bar{K^0},t;|K^0|)}
\end{equation}
Fist we define the following ket:
\begin{equation}
    \ket{K^0(t)}=\frac{\sqrt{1+|\varepsilon}|^2}{\sqrt{2}(1+\varepsilon)}(\exp{(-i\lambda_st)}\ket{K_S}+\exp{(-i\lambda_Lt)}\ket{K_L}
\end{equation}
with: $\lambda_{S/L}=m{_S/L}-\frac{i}{2}\Gamma_{S/L}$. We now compute the probabilities $P(\bar{K^0},t;|K^0|)$ and $P(K^0,t;|K^0|)$, and obtain:
\begin{equation}
    P(K^0,t;|K^0|)=\frac{1}{2|1+\varepsilon|^2}|e^{-i\lambda_S t}+
    \varepsilon e^{-i\lambda_L t}|^2
\end{equation}
and
\begin{equation}
    P(\bar{K^0},t;|K^0|)=\frac{1}{2|1-\varepsilon|^2}|
    e^{-i\lambda_S t}-\varepsilon e^{-i\lambda_L t}|^2
\end{equation}
Then:
\begin{equation}
    \delta(t)=\frac{\frac{1}{|1+\varepsilon|^2}|
    e^{-i\lambda_S t}+\varepsilon
    e^{-i\lambda_L t}|^2-\frac{1}{|1-\varepsilon|^2}|
    e^{-i\lambda_S t}-\varepsilon
    e^{-i\lambda_L t}|^2}{\frac{1}{|1+\varepsilon|^2}|
    e^{-i\lambda_S t}+\varepsilon
    e^{-i\lambda_L t}|^2+\frac{1}{|1-\varepsilon|^2}|
    e^{-i\lambda_S t}-\varepsilon
    e^{-i\lambda_L t}|^2}
\end{equation}
and with the limit of $t\rightarrow 0$ we obtain the result:
\begin{equation}
    \delta(t)=\frac{|1-\varepsilon|^2-|1+\varepsilon|^2}{|1+\varepsilon|^2+|1-\varepsilon|^2}
\end{equation}
And by expansion of the leading order in $Re(\varepsilon)$, finally:
\begin{equation}
    \frac{|1-\varepsilon|^2-|1+\varepsilon|^2}{|1+\varepsilon|^2+|1-\varepsilon|^2}=2Re(\varepsilon)+\mathcal{O}(\varepsilon^3)
\end{equation}
\subsection{Density Matrix Approach Time Evolution}
In this section we describe an open quantum system with unstable particles
(e.g. K-mesons) with the Lindbad-Gorini-Kossakowsky-Sudarhasanan master equation,
an density matrix approach, by enlarging the Hilbertspace\cite{bgh}. With this
larger Hilbertspace $\textbf{H}_{tot} = \textbf{H}_s \oplus \textbf{H}_f$ we take
into consideration both the "surviving"($\textbf{H}_s$) and the "decaying" or
"final" ($\textbf{H}_f$)
states and thus get a positive time evolution described by a non-hermitian
Hamiltonian $H_{eff}$ and a dissipator $\mathcal{D}$ of the Lindbad operator
$L$. The time evolution of the density matrix $\varrho \in \mathbf{H}_{tot}$ is given by the master
equation in the Lindbad form
\begin{align}\label{eq:master}
    \frac{d\varrho}{dt} &= -[H, \varrho] - \mathcal{D}[\varrho]\\
    \text{with}\;\;\;\;    \mathcal{D}[\varrho] &= \frac{1}{2} \sum_{j=0} (L^{\dagger}_j L_j \varrho + \varrho
    L^{\dagger}_j L_j - L_j \varrho L^{\dagger}_j)
\end{align}
where the density matrix $\varrho$ is a $4x4$ matrix with components
$\varrho_{ij}$ ($i,j = s,f$) which are $2x2$ matrices, with the property
$\varrho^\dagger_{sf} = \varrho_{fs}$
\begin{align}
    \varrho =
    \begin{pmatrix}
        \varrho_{ss} &  \varrho_{sf} \\
        \varrho_{fs} &  \varrho_{ff}
    \end{pmatrix}.
\end{align}
The Hamiltonian $H$ is an extension of the effective Hamiltonian $H_{eff}$ on
the total Hilbertspace $\textbf{H}_{tot}$
\begin{align}
    H =
    \begin{pmatrix}
        H_{eff} &  0 \\
        0 &  0
    \end{pmatrix}.
\end{align}
Furthermore the Lindbad generator $L_0$ is defined with
$B:\textbf{H}_s \rightarrow \textbf{H}_f$, where $B^\dagger B = \Gamma$, decay
matrix $\Gamma$ from the effective Hamiltonian $H_{eff}$,
\begin{align}
    L_0 =
    \begin{pmatrix}
        0 &  0 \\
        B &  0
    \end{pmatrix} \;\;\;\;
    L_j =
    \begin{pmatrix}
        A_j &  0 \\
        0 &  0
    \end{pmatrix} \;\;\;\;\;  (\text{with}\; j > 0).
\end{align}

Rewriting the master equation in \ref{eq:master} we get the following
differential equations for the density matrix components
\begin{align}
    \dot{\varrho}_{ss} &= -i[H_{eff},\varrho{ss}] - \frac{1}{2}\{B^\dagger
    B,\varrho_{ss} \} - \tilde{D}[\varrho_{ss}],\\
    \dot{\varrho}_{sf} &= -iH_{eff}\varrho_{sf} - \frac{1}{2} B^\dagger B \varrho_{sf}
    -\frac{1}{2}\sum_j A_j^\dagger A_j \varrho_{sf},\\
    \dot{\varrho}_{ff} &=B\varrho_{ss}B^\dagger .
\end{align}
with $\tilde{D}[\varrho_{ss}] = \frac{1}{2} \sum_{j=0} (A^{\dagger}_j A_j
\varrho_{ss} + \varrho_{ss}
    A^{\dagger}_j A_j - A_j \varrho_{ss} A^{\dagger}_j)$.\newline

Now we solve these equations for the case without decoherence, meaning the
Lindbad Operators operators $A_j$ disappear and we can rewrite the equations
for $\varrho_{ss}$ above in
\begin{align}
    \dot{\varrho_{ss}} &= -[H_{eff}, \varrho_{ss}] - \frac{1}{2} \{\Gamma,
    \varrho_{ss}\}=\\
    &=-i((M-\frac{i}{2}\Gamma)\varrho_{ss} - \varrho_{ss}(M-\frac{i}{2}\Gamma))
    -\frac{1}{2}(\Gamma \varrho_{ss} + \varrho_{ss}\Gamma)\\
    &= -i\underbrace{[M, \varrho_{ss}]}_{=0} - \varrho_{ss} \Gamma\\
    &= -\varrho_{ss}\Gamma \\
    \Rightarrow \;\;\; \varrho_{ss} &= \varrho_{ss}(0) e^{-\Gamma t}.
\end{align}
For $\varrho_{sf}$ we get
\begin{align}
    \dot{\varrho}_{sf} &= -i H_{eff} \varrho_{sf} - \frac{1}{2} \Gamma
    \varrho_{sf} =\\
    &= -iM\varrho_{sf}\\
    \Rightarrow \;\;\; \varrho_{sf} &= \varrho_{sf}(0) e^{-iM t}.
\end{align}
And for $\varrho_{ff}$
\begin{align}
    \dot{\varrho}_{ff} &= B\varrho_{ss}B^\dagger \\
    \Rightarrow \;\;\; \varrho_{ff}&= B\int \varrho_{ss}dt B^\dagger \\
    &= -B\varrho_{ss}(0) \Gamma^{-1} e^{-\Gamma t} B^\dagger + \varrho_{ff}(0)
\end{align}

In reality the decay rates of particles differ e.g. $K_S$ and $K_L$, the
density matrix allows such things to be taken care of by mathematically
extending the Hilbertspace and including the Lindbad operator. We could also
consider a particle with three different decay rates, though the Hamiltonian
would be a nine dimensional. In this regard we might say that the master
equation \ref{eq:master} is a more general Schrödinger equation,
because it not only describes pure quantum states but
also mixed states.

\nocite{carla}
\nocite{bgh}
\nocite{mexico}
\printbibliography
\end{document}
