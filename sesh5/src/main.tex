\documentclass[a4paper]{article}

\usepackage[T1]{fontenc}
\usepackage[utf8]{inputenc}

\usepackage{mathptmx}

\usepackage[a4paper, total={6in, 8in}]{geometry}
\usepackage{subcaption}
\usepackage[shortlabels]{enumitem}
\usepackage{bbm}
\usepackage{amssymb}
\usepackage{amsthm}
\usepackage{mathtools}
\usepackage{braket}
\usepackage{bbm}
\usepackage{graphicx}
\usepackage{float}
\usepackage{multirow}
\usepackage[colorlinks=true,naturalnames=true,plainpages=false,pdfpagelabels=true]{hyperref}
\usepackage[parfill]{parskip}
\usepackage[backend=biber, sorting=none]{biblatex}

\addbibresource{uni.bib}
\pagestyle{myheadings}

\usepackage{tikz}
\usetikzlibrary{patterns,decorations.pathmorphing,positioning}

\usepackage[framemethod=TikZ]{mdframed}

\tikzstyle{titlered} =
    [draw=black, thick, fill=white,%
        text=black, rectangle,
        right, minimum height=.7cm]

\newcounter{exercise}

\renewcommand*\theexercise{Exercise~\arabic{exercise}}

\makeatletter
\mdfdefinestyle{exercisestyle}{%
    outerlinewidth=1em,%
    outerlinecolor=white,%
    leftmargin=-1em,%
    rightmargin=-1em,%
    middlelinewidth=1.2pt,%
    roundcorner=5pt,%
    linecolor=black,%
    backgroundcolor=blue!5,
    innertopmargin=1.2\baselineskip,
    skipabove={\dimexpr0.5\baselineskip+\topskip\relax},
    skipbelow={-1em},
    needspace=3\baselineskip,
    frametitlefont=\sffamily\bfseries,
    settings={\global\stepcounter{exercise}},
    singleextra={%
        \node[titlered,xshift=1cm] at (P-|O) %
            {~\mdf@frametitlefont{\theexercise}~};},%
    firstextra={%
            \node[titlered,xshift=1cm] at (P-|O) %
                    {~\mdf@frametitlefont{\theexercise}~};},
}
\makeatother

\newenvironment{MyExercise}%
{\begin{mdframed}[style=exercisestyle]}{\end{mdframed}}





\markright{Popovic, Vogel\hfill Detection of Quantum Entanglement with MUBs \hfill}

\title{Universität Wien\\ Fakultät für Physik\\
\vspace{1.25cm}Labcours Theoretical Physik 2021S \\ Detection of Quantum
Entanglement with MUBs
}
\author{Milutin Popovic \& Tim Vogel \vspace{1cm}\\ Betreuerin: Beatrix C. Hiesmayr}
\date{20. Juni, 2021}

\begin{document}
\maketitle

\noindent\rule[0.5ex]{\linewidth}{1pt}
\begin{abstract}
    In this lab course we go through the QM problem of detecting entangled and
    separable states. Even thought given a density matrix we cannot always know if
    the state is separable or entangled. Thus a new concept is introduced, a
    witness function build upon by the so called mutually unbiased bases short
    MUBs. With help of the witness we can experimentally test to prove
    entangled Bell states.
\end{abstract}
\noindent\rule[0.5ex]{\linewidth}{1pt}

\tableofcontents

\section{Background}
\subsection{Heisenberg's uncertainty relation-Robertson version}
Given an observable $\mathcal{A}$ we can define a hermitian operator $\hat{A}$
, given a state $\psi$, we can define the expectation value $\langle \hat{A} \rangle _\psi
= \text{Tr}(\hat{A}\psi)$ and thus a standard derivation $(\Delta
\hat{A})^2_\psi = \langle \hat{A}^2 \rangle_\psi - \langle \hat{A}
\rangle_\psi^2$, where any such operator needs to satisfy.
\begin{align}
    \langle \hat{A}^\dagger \hat{A} \rangle_\psi = \langle
    \psi|\hat{A}^\dagger\hat{A} \rangle = \langle \hat{A}\psi | \hat{A}\psi
        \rangle \geq 0.
\end{align}
Furthermore two arbitrary hermitian operators $\hat{A}$ and $\hat{B}$ hold the
following inequality
\begin{align}
    (\Delta \hat{A})_\psi \cdot (\Delta \hat{B})_\psi \geq \frac{1}{2} |\langle
    \hat{A}, \hat{B}\rangle_\psi|
\end{align}
for any state $\psi$. This uncertainty is called the Heisenberg's uncertainty
principle and forms a fundamental basis for quantum mechanics the
unpredictability of quantum mechanics.

\subsection{Entropic Uncertainty Relations-Quantum Information Theoretical
Formulation}
In quantum-information theory the entropic uncertainty is defined as the
following
\begin{align}
    H(\hat{O}_n) + H(\hat{O}_m) \geq - \log_2\bigg(\max_{i,j}\{|\langle
    \chi_n^i|\chi_m^j \rangle|^2\} \bigg) = log_2(|\frac{1}{\sqrt{2}}|^2)
\end{align}
where $H(\bar{0})_n$ is the binary entropy for a pure state $\psi$
\begin{align}
    H(\bar{O}_n) = -p(n)\log_2(p(n)) -(1-p(n))\log_2(1-p(n))
\end{align}
and $p(n) = |\langle \chi_n | \psi \rangle|^2$ is the probability for the
outcome $n$ of $\hat{O}_n$ for $\psi$.

The entropic uncertainty relation can be extended for an arbitrary number of
outcomes, $d$, with the von-Neumann Entropy $S(\hat{O}_n)$
\begin{align}
    &S(\hat{O}_n) + S(\hat{O}_m) \geq - \log_d\bigg(\max_{i,j}\{|\langle
    \chi_n^i|\chi_m^j \rangle|^2\} \bigg)\\
    \nonumber\\
    &S(\hat{O}_n) = -\sum^{d-1}_{i=0} p_n(i)\ln(p_n(i)) \;\;\;\ \text{or}\\
    &S(\hat{O}_n) = -\text{Tr}(\hat{O} \ln(\hat{O})). \label{eq:vn}
\end{align}

\subsection{Mutually Unbiased Bases (MUBs)}
A ONB of a $d$-dimensional Hilbert space is $B = \{|i \rangle\} = \{|0\rangle ,\dots,
|(d-1)\rangle \}$. In quantum information theory a set of orthonormal bases
$\{B_1, \dots, B_m\}$ (each an ONB of the $d$-dimensional Hilbertspace $H^d$) is called mutually
unbiased if
\begin{align}
    |\langle i_k| j_{k'}\rangle|^2 = \delta_{k,k'}\delta_{i,j}
    (1-\delta_{k,k'})\frac{1}{d}
\end{align}

Thus the maximum of the entropy uncertainty relation is

\begin{align}
    S(\hat{O}_n) + S(\hat{O}_m) \geq - \log_d(\frac{1}{d})
\end{align}

\subsection{Construction of MUBs \label{sec:mubs}}
In this section we will show how to construct mutually unbiased bases (MUBs)
using the Hadamard Matrix $\mathbb{H}$. In fact two orthonormal basies are
connected by the Hadamard Matrix (unitary)
\begin{align}
    \mathbb{H}=\sum_{i,j} \frac{1}{\sqrt{d}} e^{i\phi_{ij}}|i\rangle\langle j|.
\end{align}
where $\phi_{i,j}$ is a phase chosen such that the $\mathbb{H}$ is unitary. A
simple choice $e^{i\phi_{i,j}} = \omega^{-ij} = e^{\frac{2\pi i}{d}}$ always
works. In this case the matrix is called the Fourier matrix
\begin{align}
    \mathbb{H}=\sum_{i,j} \frac{1}{\sqrt{d}} \omega^{-ij}|i\rangle\langle j|.
\end{align}
Furthermore the Hadamard matrix is directly related to the generalized
Pauli-matrices.
\begin{align}
    &\sigma_{\mathbb{Z}} = \sum_i \omega^i |i\rangle\langle i|\\
    &\sigma_{\mathbb{X}} = \mathbb{H}\sigma_{\mathbb{Z}} \mathbb{H} = \sum_i
    |i+1\rangle\langle i|\\
    &\sigma_{\mathbb{X}}\sigma_{\mathbb{Z}} = i \sigma_{\mathbb{Y}}.
\end{align}

All this means, the problem of finding MUBs, essentially narrows down, to finding
these Hadamad matrices.
\newline

A second way of constructing MUBs is the so called Heisenberg-Weyl
construction. If $d$ is prime, the eigenvectors of the operators, form a MUB,
which looks like:
\begin{align}
    (\sigma_\mathbb{Z},\sigma_\mathbb{X},\sigma_\mathbb{X}.
    \sigma_\mathbb{Z},\sigma_\mathbb{X}.\sigma^2_\mathbb{Z},...,\sigma_\mathbb{X}.
    \sigma^{d-1}_\mathbb{Z})
\end{align}

\textbf{Examples:}
\newline

MUBs for qubits (d=2)
\begin{align}
    &B_1 = \{|0_1\rangle, |1_1\rangle\} = \{|0\rangle, |1\rangle\}\\
    &B_2 = \{|0_2\rangle, |1_2\rangle\} = \frac{1}{\sqrt{2}} \{|0\rangle
    +|1\rangle , |0\rangle - |1\rangle\}\\
    &B_2 = \{|0_3\rangle, |1_3\rangle\} = \frac{1}{\sqrt{2}} \{|0\rangle
    +i|1\rangle , |0\rangle - i|1\rangle\}\\
\end{align}

MUBs for qutrits (d=3)
\begin{align}
    &B_1 = \{|0_1\rangle, |1_1\rangle, |2_1\rangle\} =
    \bigg\{
        \begin{pmatrix}1 \\ 0\\0\end{pmatrix},
        \begin{pmatrix}0 \\ 1\\0\end{pmatrix},
        \begin{pmatrix}0 \\ 0\\1\end{pmatrix}
    \bigg\}
    \\
    &B_2 = \{|0_2\rangle, |1_2\rangle, |2_2\rangle\} =
    \frac{1}{\sqrt{3}}
    \bigg\{
        \begin{pmatrix}1 \\ 1\\1\end{pmatrix},
        \begin{pmatrix}1 \\ \omega\\\omega^2\end{pmatrix},
        \begin{pmatrix}1 \\ \omega^2\\\omega\end{pmatrix}
    \bigg\}
    \\
    &B_3 = \{|0_3\rangle, |1_3\rangle, |2_1\rangle\} =
    \frac{1}{\sqrt{3}}
    \bigg\{
        \begin{pmatrix}1 \\\omega\\\omega\end{pmatrix},
        \begin{pmatrix}1 \\ \omega^2\\1\end{pmatrix},
        \begin{pmatrix}1 \\ 1\\\omega^2\end{pmatrix}
    \bigg\}        \\
    &B_4 = \{|0_4\rangle, |1_1\rangle, |2_1\rangle\} =
    \frac{1}{\sqrt{3}}
    \bigg\{
        \begin{pmatrix}1 \\\omega^2\\\omega^2\end{pmatrix},
        \begin{pmatrix}1 \\ \omega\\1\end{pmatrix},
        \begin{pmatrix}1 \\ 1\\\omega\end{pmatrix}
    \bigg\}
\end{align}

With these bases we can define an bell state seed $\Omega_{0,0}$ with $P_{0,0}
= |\Omega_{0,0}\rangle\langle \Omega_{0,0}|$,
\begin{align}\label{eq:arb}
    |\Omega_{0,0}\rangle  = \frac{1}{\sqrt{d}} \sum_{s=0}^{d-1} |ss\rangle
\end{align}
extending this with the Wely operators $W_{kl}$ we can arrive at an arbitrary
bell state $P_{i,j}$
\begin{align}
    &|\Omega_{k,l}\rangle = W_{kl} \otimes \mathbbm{1}|\Omega_{0,0}\rangle\\
    \nonumber\text{where:}\\
    &W_{kl} = \sum_{j=0}^{d-1} \omega^{j\cdot k} |j\rangle \langle j+l|
\end{align}
where $\omega = e^{\frac{2\pi i}{d}}$ and $\sum_{j=0}^{d-1} \omega^j = 0$.

\subsection{Detecting Entanglement via MUBs}
One of the most important aspects of quantum theory, is the prediction of
entanglement, and furthermore finding ways to construct experiments, that, with
minimal effort allow the creation of so called entanglement witnesses for
entanglement detection. Because, the bigger a system gets, the more
measurements are needed, which for huge systems is often straight up impossible
to realize. So, essentially, quantum theory tries to witness entanglement with
as few measurements as possible, and without resorting to full state
tomography.
\begin{table}[h!]
    \centering
\begin{tabular}{||c|c|c || c|c||}
\hline
    & \multicolumn{2}{|c||}{Lower Bounds} &\multicolumn{2}{|c||}{Upper Bounds}\\
\hline
    m & $L_{m,2}^{MUB}$ &$L_{m,3}^{MUB}$&$U_{m,2}^{MUB}$ &$U_{m,3}^{MUB}$ \\
\hline
    2 & 1/2 &0.211 &3/2 & 4/3\\
\hline
    3 & 1 &1/2 &2 & 5/3\\
\hline
    4 & &1 &  & 2\\
\hline
\end{tabular}
    \caption{Lower $L$ and upper $U$ bounds for the MUB witness for $d = 2, 3$
    and $m=1, \dots, d+1$ \label{tab:1}}
\end{table}

\newpage
\section{Exercises}

\begin{MyExercise}
    \textbf{Compute the Heisenberg uncertainty relation for $\hat{A} =
    \hat{\sigma}_{1}$ and $\hat{B} = \hat{\sigma}_{2}$ (Pauli matrices) for an
    arbitrary pure state $|\psi \rangle = \cos\frac{\theta}{2} |\Uparrow\rangle
    + \sin\frac{\theta}{2} e^{i\phi} |\Downarrow\rangle$. Furthermore compute
    the quantum-information theoretical version of the inequality for
    $\hat{O}_{n,m} = \hat{\sigma}_{1, 2}$}.
    \newline

    To start of, the Pauli matrices are
    \begin{align}
        \sigma_1 =
        \begin{pmatrix}
            0 & 1\\ 1& 0
        \end{pmatrix} \;\;\;\;\;
        \sigma_2 =
        \begin{pmatrix}
            0 & -i \\ i & 0
        \end{pmatrix} \;\;\;\;\;
        \sigma_3 =
        \begin{pmatrix}
            1&0\\ 0& -1
        \end{pmatrix} \;\;\;\;\;
    \end{align}
    Now we have a straight forward calculation
    \begin{align}
        &\langle \sigma_1\rangle^2_\psi = \sin^2 \theta \cos^2 \phi\\
        &\langle (\sigma_1)^2\rangle_\psi = 1\\
        \nonumber \\
        &\langle \sigma_2\rangle_\psi^2 = \sin^2\theta \sin^2\phi\\
        &\langle (\sigma_2)^2\rangle_\psi = 1\\
        \nonumber \\
        &\frac{1}{2} |\langle[\sigma_1, \sigma_2]\rangle = cos\theta
    \end{align}
    after some basic algebra with trigonometric functions we arrive at the
    following inequality
    \begin{align}
        \sin^4\theta \sin^2(2\phi) \geq 0
    \end{align}
    which holds true for all $\theta, \phi$.

    For the quantum-theoretical version of the inequality we use Equation
    \ref{eq:vn} to calculate the von Neumann entropy. The maximum of the right
    hand side is $\frac{1}{2}$
    \begin{align}
        S(\sigma_1) = -\text{Tr}(\sigma_1\ln(\sigma_1)) = 0\\
        S(\sigma_2) = -\text{Tr}(\sigma_2\ln(\sigma_2)) = \pi
    \end{align}
    thus the inequality is
    \begin{align}
        \pi \geq 1
    \end{align}

    Since the Heisenberg's uncertainty principle is mathematically correct,
    because it holds true for all hermitian operators, a violation of
    the principle would put the basis of functional analysis and/or
    the axioms of quantum mechanics at question.

    The quantum information theoretical approach to the uncertainty principal
    is convenient since the right hand side does not depend on any particular
    state.
\end{MyExercise}

\begin{MyExercise}\label{ex:2}
    \textbf{Compute
    \begin{align}
        &I_m^{MUB} = \sum_{k=1}^m\sum_{i=0}^{d-1}
        \text{Tr}((|i_k\rangle\langle i_k| \otimes |i_k\rangle \langle i_k|)
        \varrho) \;\;\;\;\;\; \text{and}\\
        &I_m^{MUB} = \sum_{k=1}^m\sum_{i=0}^{d-1}
        \text{Tr}((|i_k\rangle\langle i_k| \otimes  (|i_k\rangle \langle
        i_k|)^*)
        \varrho)
    \end{align}
    for two qubits ($d=2$), for $m=1, 2, 3$ and $|\psi\rangle =
    cos\alpha|00\rangle + sin\alpha |11>$. Here $|i_k\rangle$ is the eigenvector
    of the Pauli matrix $\sigma_k$.}

    The strategy to calculate the witness is to use the computer to loop over
    $d$ and $m$ for $m = 1, \dots, d+1$ then we compare the results with table
    \ref{tab:1}. Note that $|i_k\rangle\langle i_k|$ is a $d$-dimensional
    matrix, the density matrix is a $d^2$-dimensional matrix and
    thus the matrix inside the trace is $d^2$.

    We start of with $I_m^{MUB}$ without conjugation
    \begin{align}
        &I^{MUB}_{m=1} = \cos^2\alpha\\
        &I^{MUB}_{m=2} = \frac{1}{4}(-\sin(2\alpha) + 2\cos(2\alpha) + 3)\\
        &I^{MUB}_{m=3} = \cos^2(\alpha) + \frac{1}{2}
    \end{align}
    For $m=2$ entangled states for lower bound $\alpha = \frac{\pi}{4}$. For
    $m = 3$ entangled states for lower bound $\alpha = \frac{3\pi}{4}$.

    with conjugation we get
    \begin{align}
        &I^{MUB}_{m=1} = \cos^2\alpha\\
        &I^{MUB}_{m=2} = \frac{1}{4}(\sin(2\alpha) + 2\cos(2\alpha) + 3)\\
        &I^{MUB}_{m=3} = \frac{1}{\sqrt{2}} \sin(2\alpha + \frac{\pi}{4}) +1
    \end{align}
    For $m=2$ entangled states for lower bound $\alpha = \frac{\pi}{4}$. For
    $m = 3$ entangled states for lower bound $\alpha = -\frac{\pi}{8}$.
\end{MyExercise}

\begin{MyExercise}
    \textbf{Compute the same as in exercise \ref{ex:2} for
    the isotropic state
    \begin{align}
        \varrho^{iso}_d (p) = (1-p)\cdot\frac{1}{d^2}\mathbbm{1}_{d^2} + p
        P_{i,j}
    \end{align}
    for a freely chosen bell state $P_{i,j}$, and for both $d=2$ qubits and for
    $d=3$ qutrits. For $p\in [-\frac{1}{d^2-1}, 1]$
    we have the positivity condition and for $p\in [-\frac{1}{d^2 -1},
    \frac{1}{d+1}]$ we have a separable state else entangled.
        }

    We choose $P_{i,j} = P_{0,0} = |\Omega_{0,0}\rangle \langle \Omega_{0,0}|$.
    To calculate $\Omega$ we use the equation \ref{eq:arb} and use the MUBs given
    in section \ref{sec:mubs}.

    For $d=2$ we have the following for the standard $I^{MUB}$
    \begin{align}
        &I^{MUB}_{m=1} = \frac{1}{4}(3p+1) \\
        &I^{MUB}_{m=2} = \frac{1}{2}(3p+1) \\
        &I^{MUB}_{m=3} = \frac{1}{4}(5p+3) \\
    \end{align}
    For $m=2$ we have entanglement on the upper bound for $p = \frac{2}{3}$.
    For $m=3$ we have entanglement on the upper bound for $p = 1$.


    with conjugation we get
    \begin{align}
        &I^{MUB}_{m=1} = \frac{1}{4}(3p+1) \\
        &I^{MUB}_{m=2} = \frac{1}{2}(3p+1) \\
        &I^{MUB}_{m=3} = \frac{1}{4}(9p+3) \\
    \end{align}
    For $m=2$ we have entanglement on the upper bound for $p = \frac{2}{3}$.
    For $m=3$ we have entanglement on the upper bound for $p = \frac{4}{9}$.

    For $d=3$ we have the following
    \begin{align}
        &I^{MUB}_{m=1} = \frac{1}{9}(16p+2) \\
        &I^{MUB}_{m=2} = \frac{1}{9}(23p+4) \\
        &I^{MUB}_{m=3} = 3p + \frac{1}{3}\\
        &I^{MUB}_{m=4} = \frac{1}{9}(31p + 8)
    \end{align}
    For $m=2$ we have entanglement on the upper bound for $p = \frac{8}{23}$.
    For $m=3$ we have entanglement on the upper bound for $p = \frac{4}{9}$.

    with conjugation we get
    \begin{align}
        &I^{MUB}_{m=1} = \frac{1}{9}(16p+2) \\
        &I^{MUB}_{m=2} = \frac{1}{9}(32p+4) \\
        &I^{MUB}_{m=3} = \frac{16}{3}p + \frac{2}{3}\\
        &I^{MUB}_{m=4} = \frac{1}{9}(64p + 8)
    \end{align}
    For $m=3$ we have entanglement on the upper bound for $p = \frac{5}{3}$.
\end{MyExercise}
\newpage
\begin{MyExercise}
    \textbf{
        Compute $I_m^{MUB}$ with conjugation and without for the Werner states
        for $d=2, 3$ and $m=1,\dots, d+1$
        \begin{align}
            \varrho_W(q) = q \frac{P_{sym}}{d(d+1)} + (1-q) \frac{P_{asym}}{d(d-1)}
        \end{align}
        where $P_{sym} = (\mathbbm{1} + \mathbb{P})$ and $P_{asym} =
        (\mathbbm{1} - \mathbb{P})$  for $\mathbb{P} = \sum_{ij}
        |ji\rangle\langle ij|$. The state is separable for $q\in [0,\frac{1}{2}]$ and
        entangled for $q\in [\frac{1}{2}, 1]$
    }

    First we calculate for $d=2$ we choose the basis $B_1$ to calculate the projection
    operator. And note that $|ij\rangle = |i\rangle \otimes |j\rangle$ we need
    the tensor product here.

    Straightforward computation gives
    \begin{align}
        &I^{MUB}_{m=1} = \frac{q}{3}\\
        &I^{MUB}_{m=2} = \frac{2q}{3}\\
        &I^{MUB}_{m=3} =    q
    \end{align}
    For $m=3$ we have entanglement on the lower bound for $p = 1/2$.

    with conjugation
    \begin{align}
        &I^{MUB}_{m=1} = \frac{q}{3}\\
        &I^{MUB}_{m=2} = \frac{2q}{3}\\
        &I^{MUB}_{m=3} = \frac{q}{3} + \frac{1}{2}
    \end{align}
    For $m=3$ we have entanglement on the lower bound for $p = 0$.

    For $d=3$ we choose the basis $B_1$ to calculate the projection operator
    and straightforward computation gives
    \begin{align}
        &I^{MUB}_{m=1} = \frac{q}{3}\\
        &I^{MUB}_{m=2} = \frac{2q}{3}\\
        &I^{MUB}_{m=3} = q \\
        &I^{MUB}_{m=3} = \frac{4q}{3}
    \end{align}
    For $m=2$ we have entanglement on the lower bound for $p = 0.3165$.
    For $m=3$ we have entanglement on the lower bound for $p = \frac{1}{2}$.

    with conjugation
    \begin{align}
        &I^{MUB}_{m=1} = \frac{q}{3}\\
        &I^{MUB}_{m=2} = \frac{1}{12}(5q + 2)\\
        &I^{MUB}_{m=3} = \frac{1}{36}(15q + 14)\\
        &I^{MUB}_{m=4} = \frac{1}{36}(15q + 22)\\
    \end{align}
    For $m=2$ we have entanglement on the lower bound for $p = 0.1064$.
    For $m=3$ we have entanglement on the lower bound for $p = \frac{4}{15}$.

    A simple comparison with exercise 3, we arrive at the conclusion that for
    the Werner states we detect entanglement only on the lower bound and for the
    isotropic states we detect entanglement only on the upper bound.
\end{MyExercise}



\nocite{cite1}
\nocite{cite2}
\nocite{cite3}
\nocite{cite4}
\nocite{cite5}
\printbibliography



\end{document}
