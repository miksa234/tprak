\documentclass[a4paper]{article}

\usepackage[T1]{fontenc}
\usepackage[utf8]{inputenc}

\usepackage{mathptmx}

\usepackage[a4paper, total={6in, 8in}]{geometry}
\usepackage{subcaption}
\usepackage[shortlabels]{enumitem}
\usepackage{amssymb}
\usepackage{amsthm}
\usepackage{mathtools}
\usepackage{braket}
\usepackage{bbm}
\usepackage{graphicx}
\usepackage{float}
\usepackage[colorlinks=true,naturalnames=true,plainpages=false,pdfpagelabels=true]{hyperref}
\usepackage[parfill]{parskip}
\usepackage[backend=biber, sorting=none]{biblatex}
\addbibresource{uni.bib}
\pagestyle{myheadings}
\markright{Popovic, Vogel\hfill Kugelsymmetrische Sternenmodelle in der ART \hfill}

\title{Universität Wien\\ Fakultät für Physik\\
\vspace{1.25cm}Laborpraktikum Theoretische Physik 2021S \\ Kugelsymmetrische Sternenmodelle in der ART
}
\author{Milutin Popovic \& Tim Vogel \vspace{1cm}\\ Betreuer: Dr. Stefan
Palenta}
\date{May the 26th, 2021}

\begin{document}
\maketitle
\tableofcontents
\newpage
\section{Kugelkoordinaten in $\mathbb{R}^3$}
Im euklidischen $\mathbb{R}^3$ Raum werden die Basisvektoren mit partiellen
Ableitungen identifiziert. In kartesischen Koordinaten wird $e_x,\ e_y, e_z$
mit $\partial _x,\ \partial _y,\ \partial _z$ identifiziert. Der
Basisvektor $e_x$ z.B. gibt an in welche Richtung sich ein Punkt $P$
verschiebt, wenn man man die Koordinate $x$ um ein $dx$ vergrößert. Die
Koordinatendifferentiale sind dann $dx,\ dy,\ dz$ und die in der euklidischen
Metrik gilt dann $dx^i(\partial _j) = \partial_j x^i = \delta^i_j$.

Das selbe Spiel kann man mit Kugelkoordinaten machen
\begin{align}
    x^i =
    \begin{pmatrix}
        r\sin\theta\cos\phi\\
        r\sin\theta\sin\phi\\
        r\cos\theta
    \end{pmatrix}
\end{align}
Die Kugelkoordinaten Basisvektoren $\partial _r,\ \partial _\phi,\ \partial
_\theta$ werden durch die Kettenregel berechnet
\begin{align}
    \partial _r &= \partial _r\partial^i\partial_i \\
        &= \sin\theta\cos\phi \partial_x + \sin\theta\sin\phi \partial_y+
        \cos\theta \partial_z\\
        \nonumber\\
        \partial_\theta &= \partial_\theta  \partial^i \partial_i \\
        &=-r\cos\theta \sin\phi \partial_x + r\cos\theta \sin\phi \partial
        _y-r\sin\theta \partial_z\\
        \nonumber\\
        \partial_\phi&= \partial_\phi \partial^i \partial_i \\
        &=-r\sin\theta\sin\phi\partial_x + r\sin\theta\cos\phi \partial_y
\end{align}
Die Einheitsvektoren sind die normierten Basisvektoren
\begin{align}
    e_r = \partial_r;\;\;\; e_\theta = \frac{1}{r}\partial_\theta;\;\;\;
    e_\phi = \frac{1}{r\sin\phi}\partial _\phi.
\end{align}
Wichtiger einschub ist, dass die Lie-Klammer $[e_r, e_\phi]$ nicht
verschwindet.
\begin{align}
    [e_r, e_\theta]&=e_r e_\theta - e_\theta e_r = \partial_r
    \frac{1}{r}\partial_\theta  - \frac{1}{r}\partial_\theta\partial_r =\\
    &=-\frac{1}{r^2}\partial_\theta + \frac{1}{r} \partial_r\partial_\theta
    -\frac{1}{r}\partial_\theta\partial_r=\\
    &=\frac{1}{r}([\partial_r, \partial_\theta] - \partial_\theta)
\end{align}
was bedeutet, dass $e_r$ nicht mit $e_\theta$ kommutiert. Damit die
Einheitsvektoren eine Koordinatenbasis bilden können müssen sie linear
unabhängig voneinander sein, da sie aber eine nicht-triviale Lie-Klammer
besitzen, sind sie nicht linear unabhängig und können somit keine
Koordinatenbasis sein kann.

Die Metrik $g_{ij}$ in Kugelkoordinaten ist verschwindend für alle
$i\neq j$, sie kann sie ausgerechnet werden durch z.B. $g_{rr}
=\bar{g}(\partial_r, \partial_r)$.
\begin{align}
    g_{rr} &= (\sin\theta\cos\phi e_x  + \sin\theta\sin\phi e_y + \cos\theta e_z)^2\\
    &= 1\\
    \nonumber\\
    g_{\phi\phi} &= (-r\sin\theta\sin\phi e_x + r\cos\phi\sin\phi
    e_y)^2=\\
    &=r^2\sin^2\theta\sin^2\phi + r^2\sin^2\phi \\
    &=r^2 \sin^2\theta\\
    \nonumber\\
    g_{\theta\theta} &= (r\cos\theta\sin\phi e_x + r\cos\phi\sin\phi e_\phi -
    r\sin e_z)^2 \\
    &= r^2\\
    \nonumber\\
    (g_{ij}) &=
    \begin{pmatrix}
        1&0&0\\
        0&r^2&0\\
        0&0&r^2\sin^2\theta
    \end{pmatrix}
\end{align}
Weiteres können wir die kovariante Ableitung $\nabla_{\partial _a}$ (kurz
$\nabla _a$ eines
Vektorfeldes $X = X^b\partial _b$ entlang
$\partial _a$ betrachten, dabei tauchen die Cristoffelsymbole
$\Gamma^c_{ab}$ auf.
\begin{align}
    \nabla_{\partial_a}(X^b\partial _b) &= (\nabla_{\partial_b}X^b)\partial_b
    + X^b(\nabla_{\partial_a}\partial_b) =\\
    &= (\nabla_{\partial_b}X^b)\partial_b + X^b\Gamma^c_{ab} \partial_c
\end{align}
Die Christoffelsymbole sind gegeben durch die Metrik-Komponenten.
\begin{align}
    \Gamma^{c}_{ab} = \frac{1}{2} g^{ce}(\partial_a g_{cb} + \partial_b
    g_{ac} - \partial_c g_{ab}).
\end{align}
Offensichtlich verschwinden die Christoffelsymbole bzüglich der karthesischen
Koordinatenbasis, da $g_{ij} = \delta_{ij}$ konstant ist. Nun berechnen wir
die Crhistoffelsymbole $\Gamma^{d}_{\theta\phi}$ und $\Gamma^{d}_{\phi\phi}$
bezüglich den Kugelkoordinaten ($d \in \{r, \theta, \phi\}$).
\begin{align}
\Gamma^{r}_{\theta\phi}&=\frac{1}{2}g^{rr}(\partial_\theta g_{\phi r}
+\partial_\phi g_{\theta r} - \partial_r g_{\phi\theta}) = 0\\
    \Gamma^{r}_{\phi\phi} &=-\frac{1}{2}g^{rr}\partial_r g_{\phi\phi}=
    -r\sin^\theta\\
    \Gamma^{\theta}_{\theta\phi} &= \frac{1}{2}g^{\theta\theta}\partial_\phi
    g_{\theta\theta} = 0\\
    \Gamma^{\theta}_{\phi\phi} &= -\frac{1}{2}g^{\theta\theta}\partial_\theta
    g_{\phi\phi} = -2\sin2\theta\\
    \Gamma^{\phi}_{\theta\phi} &= \frac{1}{2}g^{\phi\phi}\partial_\theta
    g_{\phi\phi} = \cot\theta\\
    \Gamma^{\phi}_{\phi\phi} &= \frac{1}{2}g^{\phi\phi}\partial_\phi
    g_{\phi\phi} = 0.
\end{align}
Die Christoffelsymbole sind symmetrisch bezüglich der Vertauschung der
kovarianten Indizes, d.h.
\begin{align}
    \Gamma^r_{\phi\theta} &= 0\\
    \Gamma^\theta_{\phi\theta} &= 0 \\
    \Gamma^\phi_{\phi\theta} & = \frac{1}{2} g^{\phi\phi} \partial_\theta
    g_{\phi\phi} = \cot\phi
\end{align}
\section{Differentialoperatoren}
In der Allgemeinen Relativitätstheorie ist eine allgemeine Metrik gegeben,
eine symmetrische $n\ x\ n$ Matrix $g_{ab}$. Mit der inversen Metrik $g^{ab}$
ergibt sich die triviale Identität $g_{ab}g^{bc} = \delta ^c_a$. Mithilfe der
Determinante $g := \det(g_{ab})$ und der Cramer'schen Regel
kann auf die inverse Matrix umgeformt
werden.
\begin{align}
    g = g_{ab}\ \text{adj}(g_{ab})
\end{align}
Differentiert man diese Gleichung auf beiden Seiten mit
$\frac{\partial}{\partial g_{ij}}$
\begin{align}
    &\frac{\partial g }{\partial g_{ij}} = \delta_{ab}^{ij}\
    \text{adj}(g_{ab}) = \text{adj}(g_{ij}) = g^{ab} \cdot g \\
    \nonumber \\
    &\Rightarrow \frac{1}{g} \frac{\partial g}{\partial g_{ij}} =
    g^{ij}.
\end{align}

Als nächstes zeigen wir eine Relation für Christoffelsymbole $\Gamma
^{\mu}_{\mu\nu}$ bezüglich einer allgemeinen Metrik.

\begin{align}
    \Gamma^{\mu}_{\mu\nu} &= \frac{1}{2} g^{\mu\varrho}(\partial _\nu
    g_{\mu\nu} +\partial_\mu g_{\nu\varrho} - \partial _\varrho
    g_{\mu\nu})=\\
    &= \frac{1}{2} g^{\mu\varrho} \partial_\nu g_{\mu\varrho}.
\end{align}
Betrachtet man die Ableitung von $g$ nach $\partial _\nu$ bekommt man
\begin{align}
    &\partial _\nu g = g g^{\mu\varrho} \partial_nu g_{\mu\varrho} \\
    &\Rightarrow \frac{\partial _\nu g}{g} = g^{\mu\varrho}\partial_\nu
    g_{\mu\varrho}.
\end{align}
Man kann die wurzel von $g$, $\sqrt{g}$ betrachten dann kommt ein Faktor von
$\frac{1}{2}$ durch die Kettenregel und es ergibt sich die allgemeine
Relation
\begin{align}
    \frac{1}{\sqrt{g}} \partial_\nu \sqrt{g} = g^{\mu\varrho}
    \partial _\nu g_{\mu\varrho} = \Gamma ^{\mu}_{\mu\nu}.
\end{align}
Weiteres zeigen wir eine weitere Relation zur kovarianten Ableitung eines
Vektorfeldes $\nabla_a A^a$
\begin{align}
\nabla_a A^a = \frac{1}{\sqrt{g}} \partial_a (\sqrt{g} A^a).
\end{align}
Nun mithilfe von dem Levi-Civita-Zusammenhang
\begin{align}
    \nabla _a A^a = \partial _a A^a + \Gamma^a_{ac} A^a
\end{align}
um das Christoffelsymbol auszurechnen benutzen wir die allgemein gültige
Relation
\begin{align}
    \nabla _a A^a &= \partial _a A^a + (\frac{1}{\sqrt{g}}) \partial _a
    \sqrt{g}) A^a = \\
    &= \partial _a A^a + \frac{1}{\sqrt{g}}\partial_a(\sqrt{g} A^a) -
    \frac{\sqrt{g}}{\sqrt{g}} \partial _a A^a = \\
    &= \frac{1}{\sqrt{g}} \partial _a ( \sqrt{g} A^a)
\end{align}
Das selbe kann man mit eine antisymmetrischen $(2, 0)$ Tensor $F^{ab} = -
F^{ba}$
\begin{align}
    \nabla _{a} F^{ab} = \partial_a F^{ab} + \Gamma^{a}_{ac} F^{cb}
    \Gamma^{b}_{ac} F^{ac}
\end{align}
wobei hier das letztere Christoffelsymbol $\Gamma^{b}_{ac}$ verschwindet
wegen der antisymmetrie des Tensors $F^{ac}$. Weiterhin schreiben wir
wieder die allgemeine Relation für $\Gamma ^{a}_{ac}$ und wenden
die umgekehrte Produktregel an, somit kommen wir auf
\begin{align}
    \nabla_a F^{ab} =  \frac{1}{\sqrt{g}} \partial_c(\sqrt{g}F^{ca})
\end{align}
In der Elektrodyamik ist $F^{ab}$ der Maxwelltensor und die oberen
Gleichungen representieren die Maxwellgleichungen.
\begin{align}
    \nabla_a F^{ab} = J^b
\end{align}
Die kontinuitätsgleichung $\nabla_b J^b = 0$ kann leicht gezeigt werden
\begin{align}
    \nabla_b \nabla_a F^{ab} &= \nabla_b(\frac{1}{\sqrt{g}}
    \partial_a(\sqrt{g}F^{ab})) =\\
    &= \partial_b \frac{1}{\sqrt{g}} \partial_a(\sqrt{g}F^{ab})+
    \frac{1}{\sqrt{g}}
    \partial_b (\sqrt{g}\frac{1}{\sqrt{g}}\partial_a(\sqrt{g} F^{ab}))=\\
    &=-\frac{1}{2}\frac{1}{\sqrt{g}}\partial_a F^{ab}\partial_b
    \sqrt{g} + \frac{1}{2} \frac{1}{\sqrt{g}} \partial_a F^{ab} \partial_b
    \sqrt{g} \\
    &= 0
\end{align}
Weiterhin zeigen wir eine weitere Relation, dabei wenden wir zwei mal die
kovariante Ableitung auf ein Skalarfeld $U$.
\begin{align}
    g^{ab}\nabla_a\nabla_b U &= \Delta U = \nabla_a \partial^a U =\\
    &= \partial_a \partial^a U + \Gamma^{b}_{ac} \partial^a U =\\
    &= \partial_a\partial^a U
    \frac{1}{\sqrt{g}}\partial_a(\sqrt{g}\partial^a U) = \\
    &= \frac{1}{\sqrt{g}} \partial_a(\sqrt{g} \partial^a U)
\end{align}

\section{Herleitung der TOV-Gleichung}
Wir können das Innere eines Stern näherungsweise als ideale Flüssigkeit betrachten. In diesem Fall lautet der Energie-Impuls-Tensor ebendieser Flüssigkeit:
\begin{equation}
    T_{ab}=(\mu+\frac{p}{c^2})u_au_b+pg_{ab}
\end{equation}
In Schwarzschildkoordinaten, und mittels $u^au_a=-c^2$, lassen sich die Komponenten dieses Tensors wie folgt berechnen:
\begin{equation}
    T_{00}=(\mu+\frac{p}{c^2})u_0u_0-pg_{00}=(\mu+\frac{p}{c^2})c^2e^{\mu(r)}-pe^{\mu(r)}=\mu c^2e^{\mu(r)}
\end{equation}
ähnlich erhalten wir für die restlichen Diagonalelemente:
\begin{equation}
    T_{11}=pe^{\lambda(r)}
\end{equation}
\begin{equation}
    T_{22}=pr^2
\end{equation}
\begin{equation}
    T_{33}=pr^2\sin^2{\theta}
\end{equation}
Dies bedeutet, der Energie-Impuls-Tensor, nimmt folgende Form an:
\begin{equation}
    T_{ab}=diag(\mu c^2e^{\nu(r)},pe^{\lambda(r)},pr^2,pr^2\sin{\theta})^t
\end{equation}
Wir betrachten nun die Einsteinschen Feldgleichungen:
\begin{equation}
    R_{ab}-\frac{1}{2}Rg_{ab}=\kappa T_{ab}
\end{equation}
Diese, können mitthilfe von $T=-\frac{R}{\kappa}$ umgeformt werden, zu:
\begin{equation}
    R_{ab}=\kappa(T_{ab}-\frac{1}{2}Tg_{ab})
\end{equation}
Wir wollen nun die Komponenten von Gleichung 127 berechnen, die ungleich Null sind.
Die Komponenten von $T^\mu_\nu$ erhalten wir durch Verjüngung des Energie-Impuls-Tensors mir der Metrik $g^{\mu\nu}$.
Diese ergeben sich zu:
\begin{equation}
    T^0_0=(\mu+\frac{p}{c^2})u^0u_0+pg^0_0=-\mu c^2
\end{equation}
\begin{equation}
    T^1_1=p
\end{equation}
\begin{equation}
    T^2_2=p
\end{equation}
\begin{equation}
    T^3_3=p
\end{equation}
Daraus folgt, für $R_{ab}$:
\begin{equation}
    R_{00}=\kappa(\mu c^2e^{\nu(r)}-\frac{1}{2}\mu c^2e^{\nu(r)})=\kappa\frac{1}{2}\mu c^2e^{\nu(r)}
\end{equation}
\begin{equation}
    R_{11}=\kappa\frac{1}{2}pe^{\lambda(r)}
\end{equation}
\begin{equation}
    R_{22}=\kappa\frac{1}{2}pr^2
\end{equation}
\begin{equation}
    R_{33}=\kappa\frac{1}{2}r^2\sin^2{\theta}
\end{equation}



\section{Inkompressibler Relativistischer Stern}
Wir berechnen nun der Druckverlauf eines relativistischen Sterns mit
konstanter Massendichte $\mu$ mithilfe der TOV-Gleichung. Der Druck an der
Sternoberfläche ist $p(r_0) =0$. In der Rechnung substituieren wir
$P = \kappa p$, $A = \frac{8\pi\mu}{3c^2}$ und
$x=r^2 \Rightarrow dr= \frac{dx}{2r}$. Weiters ist zu
beachten, das $\mu = \text{konst.}$ und somit
die Masse des Sterns gegeben ist durch
\begin{align}
    m(r) = m = 4pi \int_0^r \mu \tilde{r}^2 d\tilde{r} =\frac{4\pi}{3}\mu r^3
\end{align}
Wir kommen auf die folgende Differentialgleichung.
\begin{align}
    \frac{dP}{dx} = -\frac{(3A+P)(A+P)}{(1-Ax)}.
\end{align}
Diese lässt sich durch Separation der Variablen lösen und mithilfe von
$p(r_0) = 0$ bekommen wir die Lösung
\begin{align}
    P(r) = 3A\left( \frac{\sqrt{\frac{1-Ar^2}{1-Ar_0^2}} -
    1}{3-\sqrt{\frac{1-Ar^2}{1-Ar_0^2}}}\right)
\end{align}
Der Zentraldruch $p_c$ ist gegeben durch
\begin{align}
    p_c = p(0) = \frac{3A}{\kappa}\left( \frac{\sqrt{\frac{1}{1-Ar_0^2}} -
    1}{3-\sqrt{\frac{1}{1-Ar_0^2}}}\right)
\end{align}
Weiteres aus $e^{\lambda(r)} > 0$ folgt $e^{\lambda(r_0)} > 0$, was ergibit
\begin{align}
    &\sqrt{1-Ar_0^2}>\frac{1}{3}\\
    &\Rightarrow r_0 > \sqrt{\frac{8}{9A}}
\end{align}
hierraus leiten wir die obere Schranke der Masse ab, da $\mu=\text{konst.}$
rechnet sich $M$ wie oben, $M = \frac{4\pi}{3} \mu r_0^3$. Wir qubieren die
ungleichung von oben und setzen für $A$ ein, es ergibt sich
\begin{align}
    M < M_{max} = \frac{4c^3}{9\sqrt{3\pi G^3\mu}}
\end{align}
Die beiden Metrischen Funktionen lassen sich leicht durch einsetzen
herausfinden
\begin{align}
    &e^{\lambda(r)} = \frac{1}{1 - Ar^2}\\
    &e^{\nu(r)} = e^{-\lambda(r)} = 1 - Ar^2
\end{align}
Aus $Ar_0^2 = \frac{r_s}{r_0}$ ergibt sich die Buchdahl Grenze
\begin{align}
    Ar_0^2 = \frac{r_s}{r_0} < \frac{8}{9}
\end{align}
Für den gravitative Massendeffekt berechnen wir $\Delta M= M_1 -M$, wobei
$M_1$ durhc das folgende Integral gegeben ist.
\begin{align}
    M_1 &= 4\pi \int _0^{r_0}\mu(r) e^{\lambda(r)/2}r^2 dr =\\
    &= \frac{4\pi\mu} \int_0^{r_0} \sqrt{\frac{1}{1-Ar^2}} r^2 dr= \\
    &= \frac{2\pi\mu}{A^{3/2}} (\arcsin(\sqrt{A}r_0) -Ar_0\sqrt{1-Ar_0^2})
\end{align}
Dann ist $\Delta M$
\begin{align}
    \Delta M \frac{2\pi\mu}{A^{3/2}} (\arcsin(\sqrt{A}r_0)
    -Ar_0\sqrt{1-Ar_0^2})  + \frac{4\pi}{3} \mu r_0^3
\end{align}
Der Newtonische Grenzfall der gravitativen Bindungsenergie ist
\begin{align}
    W &= -\Delta M c^2 = -c^2(4\pi\int _0^{r_0} \frac{1}{\sqrt{1-Ar^2}} r^2
    dr + \frac{4\pi}{3} \mu r^3_0) \\
    \lim_{c\rightarrow \infty} W &= -4\pi G \int_0^{r_0} m rdr
    = 2\pi Gmr_0^2
\end{align}

\section{Effektive Berechnungsmethode für Christoffelsymbole}
In einer gekrümmten Raumzeit, lassen sich Geodäten, also die kürzeste Verbindung zweier Punkte innerhalb dieser Raumzeit, mithilfe der Geodäten-Gleichung berechenen:
\begin{equation}
    \frac{d^2x^\alpha}{d\sigma^2}+\Gamma^{a}_{bc}\frac{dx^b}{d\sigma}\frac{dx^c}{d\sigma}
\end{equation}
Hierbei ist $\sigma$ ein affiner Paramter, der mit der Weglänge zusammenhängt. Die CHristoffel-Symbole in der obigen Gleichung, lassen sich mittels einer Lagrange-Funktion, relativ zügig ermitteln. Wir stellen hierfür die Lagrange-Funktion auf, als:
\begin{equation}
    L=g_{\rho\sigma}(x^n)\dot{x}^\rho\dot{x}^\sigma
\end{equation}
Mithilfe dieser Gleichungen, folgt die Äquivalenz:
\begin{equation}
    g_{\alpha\beta}\ddot{x^\beta}+\Gamma_{\alpha ab}\dot{x^a}\dot{x^b}=\frac{d}{d\tau}(\frac{\partial L}{\partial\dot{x^\alpha}})-\frac{\partial L}{\partial x^\alpha}
\end{equation}
 Mithilfe der Metrik
 \begin{equation}
     \begin{pmatrix}
         1 & 0 & 0 \\
         0 & r^2 & 0 \\
         0 & 0 & r^2\sin^2{\theta}
     \end{pmatrix}
     =(g_{ab})
 \end{equation}
folgt für die Lagrange-Funktion nun:
\begin{equation}
    L=\frac{1}{2}(\dot{r^2}+r^2\dot{\theta^2}+r^2\sin^2{\theta}\dot{\phi^2}
\end{equation}
Wir berechnen nun die Ableitungen von L:
\begin{equation}
    \frac{d}{d\tau}\frac{\partial L}{\partial\dot{r}}=\ddot{r}
\end{equation}
\begin{equation}
    \frac{d}{d\tau}\frac{\partial L}{\partial\dot{\theta}}=2\dot{r}\dot{\theta}+r^2\ddot{\theta}
\end{equation}
und
\begin{equation}
    \frac{d}{d\tau}\frac{\partial L}{\partial\dot{\phi}}=2\dot{r}r\sin^2{\theta}\dot{\phi}+2r^2\sin{\theta}\cos{\theta}\dot{\theta}\dot{\phi}+r^2\sin^2{\theta}\ddot{\phi}
\end{equation}
Des Weiteren:
\begin{equation}
    \frac{\partial L}{\partial r}=r\dot{\theta^2}+r\sin^2{^\theta}\dot{\phi^2}
\end{equation}
\begin{equation}
    \frac{\partial L}{\partial\phi}=0
\end{equation}
\begin{equation}
    \frac{\partial L}{\partial\theta}=r^2\sin{\theta}\cos{\theta}\dot{\phi^2}
\end{equation}
Aus diesen Gleichungen kann man nun die Christoffel-Symbole bestimmen, und diese ergeben sich zu:
\begin{equation}
    \Gamma_{r\theta\theta}=-r
\end{equation}
\begin{equation}
    \Gamma_{r\phi\phi}=-r\sin^2{\theta}
\end{equation}
\begin{equation}
    \Gamma_{\theta r\theta}=2
\end{equation}
\begin{equation}
    \Gamma_{\theta\phi\phi}=-r^2\sin{\theta}\cos{\theta}
\end{equation}
\begin{equation}
    \Gamma_{\phi r\phi}=2r\sin^2{\theta}
\end{equation}
und
\begin{equation}
    \Gamma_{\phi\theta\phi}=2r^2\sin{\theta}\cos{\theta}
\end{equation}
\newline

\section{Schwarzschild Metic}
Eine Kugelsymmetrische Raumzeit kann durch die Schwarzschild-Koordinaten
beschrieben werden $\{ct, r, \vartheta \varphi\}$, für das Linienelemnt haben
wir den folgenden Ansatz gegeben
\begin{align}
    ds^2 = -e^\nu c^2 d^2t + e^\lambda d^2r + r^2(d^2\vartheta
    \sin^2\vartheta d^2\varphi).
\end{align}
wobei hier $\nu = \nu(t, r)$ und $\lambda = \lambda(t, r)$, Funktionen sind
die noch bestimmt werden. Mithilfe dieses Linienelements
können wir direkt die Metrik ablesen. Weiterhin setzen wir $c=1$ und
schreiben
\begin{align}
    &ds^2 = g_{00} dt^2 + g_{11} dr^2 + g_{22} d^2\vartheta + g_{33}
    d^2\varphi
    \nonumber \\
    &\Rightarrow (g_{ij}) =
    \begin{pmatrix}
        -e^\nu & 0 & 0 & 0  \\
        0 & e^\lambda & 0 & 0  \\
        0 & 0 & r^2 & 0  \\
        0 & 0 & 0 & r^2\sin^2\vartheta  \\
    \end{pmatrix}
\end{align}
Alle von null verschiedenen Christoffelsymbole lassen sich leicht berechnen,
da die meisten Koeffizienten wegfallen aufgrund der diagonalen Metrik
\begin{align}
    \Gamma^{0}_{00} &= \frac{1}{2} g^{00}\partial_0 g_{00} = \frac{1}{2}
    \dot{\nu}\\
    \Gamma^{0}_{01} &= \frac{1}{2} g^{00}\partial_1 g_{00} = \frac{1}{2}
    \nu'\\
    \Gamma^{0}_{11} &= -\frac{1}{2} g^{00}\partial_0 g_{11} = \frac{1}{2}
    e^{\lambda - \nu} \dot{\lambda}\\
    \Gamma^{1}_{00} &= -\frac{1}{2} g^{11}\partial_1 g_{00} = \frac{1}{2}
    e^{\nu -\lambda} \nu'\\
    \Gamma^{1}_{10} &= \frac{1}{2} g^{11}\partial_0 g_{11} =
    \frac{\dot{\lambda}}{2}\\
    \Gamma^1_{22} &= -\frac{1}{2} g^{11} \partial_1 g_{22} = -e^{-\lambda}
    r\\
    \Gamma^{2}_{12} &= \frac{1}{2} g^{22} \partial_1 g_{22} = r\\
    \Gamma^{2}_{33} &= \frac{1}{2} g^{22} \partial_2 g_{33} =
    \frac{1}{2}\sin(2\vartheta)\\
    \Gamma^{3}_{11} &= \frac{1}{2} g^{33} \partial_1 g_{33} = \frac{1}{r}\\
    \Gamma^{3}_{23} &= \frac{1}{2} g^{33}\partial_2 g_{33} = \cot\vartheta
\end{align}
Nun können wir den Ricci Tensor $R_{ab}$ ausrechnen
\begin{align}
    R_{ab}:=R^c_{abc}  = \Gamma^c_{ab,c}-\Gamma^c_{ac, b} + \Gamma^c_{ab}
    \Gamma^d_{ad} -\Gamma^c_{db}\Gamma^d_{ac}.
\end{align}
Für die Indizes $00$ zeigen wir
\begin{align}
    R_{00} &= \Gamma^c_{00,c} - \Gamma^c_{0c,0}+\Gamma^c_{dc}\Gamma^d_{00} -
    \Gamma^c_{d0}\Gamma^d_{0c} =\\
    &=\Gamma^0_{00,0} + \Gamma^1_{00,1} - \Gamma^0_{00} + \Gamma^1_{01,0}
    \Gamma^c_{0c}\Gamma^{0}_{00} - \Gamma^c_{00}\Gamma^0_{0c} - \Gamma^c_{10}
    \Gamma^1_{0c} =\\
    &= \Gamma^1_{00,1} - \Gamma^1_{01,0} + \Gamma^1_{01}\Gamma^0_{00}
    \Gamma^1_{11}\Gamma^1_{00} - \Gamma^1_{00}\Gamma^0_{01} - \Gamma^1_{10}
    \Gamma^1_{01} = \\
    &=\frac{1}{2}e^{\lambda-\nu}(\frac{\nu'}{r} + \nu'')
    -\frac{1}{2}\ddot{\lambda} + \frac{1}{4} \dot{\lambda}\dot{\nu}
    \frac{1}{4}e^{\lambda-\nu} \nu'\lambda' - \frac{1}{4}e^{\lambda-\nu}
    \nu'\nu' - \frac{1}{4} \dot{\lambda}\dot{\lambda} = \\
    &=\frac{1}{2} e^{\lambda -\nu} ( \nu'' -\frac{1}{2}(\lambda' - \nu')
    \frac{\nu'}{r}) -\frac{1}{2} \ddot{\lambda} + \frac{1}{4}\dot{\lambda}(
    \dot{\nu} - \dot{\lambda}).
\end{align}
Die anderen nich-trivialen Koeffizienten lauten
\begin{align}
    R_{11} &= e^{\lambda-\nu}
    (\frac{1}{2}\ddot{\lambda}+\frac{1}{4}\dot{\lambda}(\dot{\lambda}-
    \dot{\nu}))
    -\frac{1}{4} \nu'(\lambda' - \nu') + \frac{\lambda'}{r}\\
    R_{22} &= 1-e^{-\lambda}(1+\frac{1}{2r} (\nu'-\lambda'))\\
    R_{23} &= \sin^2\vartheta R_{22}\\
    R_{01} &= R_{10} = \frac{\dot{\lambda}}{r}.
\end{align}
Im Vakuum besagen die Einsteinschen Feldgleichungen, dass $R_{ab} = 0$ für
alle $a, b$, somit folgt sofort aus $R_{01} = 0$ dass $\lambda = \lambda(r)$.
Die Funktion $\lambda$ hängt nur vom Abstand ab. Differenziert man $R_{22}$
nach der $0$ Koordinate so erhält man ein Ergebnis für $\nu$
\begin{align}
    &\partial_0 R_{22} = -e^{-\lambda} \dot{\lambda} ( 1+ \frac{r}{2}(\nu' -
    \lambda')) + e^{-\lambda} \frac{r}{2}\partial_0 \nu' = 0\\
    \nonumber\\
    &\Rightarrow \partial_0\partial_1 \nu = \partial_1\partial_0 = 0\\
    &\Rightarrow \partial_0 \nu =0 \Leftrightarrow \nu = \nu(r)
\end{align}
weiterhin zeigen wir, dass $\lambda = -\nu$ indem wir folgendes rechnen
\begin{align}
    &R_{00} + e^{\lambda-\nu} R_11 = 0\\
    &\frac{\lambda'}{r} + \frac{\nu'}{r} = 0 \Rightarrow \lambda = -\nu.
\end{align}
Aus $R_{22} = 0$ und der obigen Relation lässt sich eine
Differentialgleichung für $\nu$ aufstellen
\begin{align}
    R_{22} &= 1 - e^\nu (1 + \frac{1}{2}r(\nu'+ \nu')) = \\
    &= 1 - e^\nu (1+ \nu' r) = 0.
\end{align}
Diese kann man leicht durch Separation der Variablen lösen
\begin{align}
    &\int \frac{d\nu}{e^{-\nu} -1} = \int \frac{1}{r} dr\\
    &\Rightarrow e^\nu = 1-\frac{r_S}{r} \;\;\;\; (r_S\in \mathbb{R})
\end{align}
<<<<<<< HEAD
\section{Inkompressibler Relativistischer Stern}
Wir berechnen nun der Druckverlauf eines relativistischen Sterns mit
konstanter Massendichte $\mu$ mithilfe der TOV-Gleichung. Der Druck an der
Sternoberfläche ist $p(r_0) =0$. In der Rechnung substituieren wir
$P = \kappa p$, $A = \frac{8\pi\mu}{3c^2}$ und
$x=r^2 \Rightarrow dr= \frac{dx}{2r}$. Weiters ist zu
beachten, das $\mu = \text{konst.}$ und somit
die Masse des Sterns gegeben ist durch
\begin{align}
    m(r) = m = 4pi \int_0^r \mu \tilde{r}^2 d\tilde{r} =\frac{4\pi}{3}\mu r^3
\end{align}
Wir kommen auf die folgende Differentialgleichung.
\begin{align}
    \frac{dP}{dx} = -\frac{(3A+P)(A+P)}{(1-Ax)}.
\end{align}
Diese lässt sich durch Separation der Variablen lösen und mithilfe von
$p(r_0) = 0$ bekommen wir die Lösung
\begin{align}
    P(r) = 3A\left( \frac{\sqrt{\frac{1-Ar^2}{1-Ar_0^2}} -
    1}{3-\sqrt{\frac{1-Ar^2}{1-Ar_0^2}}}\right)
\end{align}
Der Zentraldruch $p_c$ ist gegeben durch
\begin{align}
    p_c = p(0) = \frac{3A}{\kappa}\left( \frac{\sqrt{\frac{1}{1-Ar_0^2}} -
    1}{3-\sqrt{\frac{1}{1-Ar_0^2}}}\right)
\end{align}
Weiteres aus $e^{\lambda(r)} > 0$ folgt $e^{\lambda(r_0)} > 0$, was ergibit
\begin{align}
    &\sqrt{1-Ar_0^2}>\frac{1}{3}\\
    &\Rightarrow r_0 > \sqrt{\frac{8}{9A}}
\end{align}
hierraus leiten wir die obere Schranke der Masse ab, da $\mu=\text{konst.}$
rechnet sich $M$ wie oben, $M = \frac{4\pi}{3} \mu r_0^3$. Wir qubieren die
ungleichung von oben und setzen für $A$ ein, es ergibt sich
\begin{align}
    M < M_{max} = \frac{4c^3}{9\sqrt{3\pi G^3\mu}}
\end{align}
Die beiden Metrischen Funktionen lassen sich leicht durch einsetzen
herausfinden
\begin{align}
    &e^{\lambda(r)} = \frac{1}{1 - Ar^2}\\
    &e^{\nu(r)} = e^{-\lambda(r)} = 1 - Ar^2
\end{align}
Aus $Ar_0^2 = \frac{r_s}{r_0}$ ergibt sich die Buchdahl Grenze
\begin{align}
    Ar_0^2 = \frac{r_s}{r_0} < \frac{8}{9}
\end{align}
Für den gravitative Massendeffekt berechnen wir $\Delta M= M_1 -M$, wobei
$M_1$ durhc das folgende Integral gegeben ist.
\begin{align}
    M_1 &= 4\pi \int _0^{r_0}\mu(r) e^{\lambda(r)/2}r^2 dr =\\
    &= \frac{4\pi\mu} \int_0^{r_0} \sqrt{\frac{1}{1-Ar^2}} r^2 dr= \\
    &= \frac{2\pi\mu}{A^{3/2}} (\arcsin(\sqrt{A}r_0) -Ar_0\sqrt{1-Ar_0^2})
\end{align}
Dann ist $\Delta M$
\begin{align}
    \Delta M \frac{2\pi\mu}{A^{3/2}} (\arcsin(\sqrt{A}r_0)
    -Ar_0\sqrt{1-Ar_0^2})  + \frac{4\pi}{3} \mu r_0^3
\end{align}
Der Newtonische Grenzfall der gravitativen Bindungsenergie ist
\begin{align}
    W &= -\Delta M c^2 = -c^2(4\pi\int _0^{r_0} \frac{1}{\sqrt{1-Ar^2}} r^2
    dr + \frac{4\pi}{3} \mu r^3_0) \\
    \lim_{c\rightarrow \infty} W &= -4\pi G \int_0^{r_0} m rdr
    = 2\pi Gmr_0^2
\end{align}
=======
\section{Inkompressibler Newton'scher Stern}
Das Gravitationspotential $U$ eines Sterns, mit Radius $r_0$ und konstanter Massendichte $\mu$, im Rahmen der Newton'schon Physik, kann leicht mittels der Poissongleichung ermittelt werden. Diese lautet:
\begin{equation}
    \Delta U=4\pi G\mu
\end{equation}
Um das Potential zu ermittlen, betrachten wir zunächst die Lösung der Poissongleichung außerhalb des Sterns. Diese redzuiert sich somit zu:
\begin{equation}
    \Delta U=0
\end{equation}
oder in Kugelkoordinaten:
\begin{equation}
    \Delta U=\frac{1}{r^2}\frac{\partial}{\partial r}(r^2\frac{\partial U}{\partial r}+\frac{1}{r^2\sin{\theta}}\frac{\partial}{\partial\theta}(\sin{\theta}\frac{\partial U}{\partial\theta}+\frac{1}{r^2\sin^2{\theta}}\frac{\partial^2U}{\partial\phi^2}=0
\end{equation}
Es wird sofort klar, dass U nicht von $\theta$ und $\phi$ abhängen kann, und somit wird das Potential zu:
\begin{equation}
    \Delta U=\frac{1}{r^2}\frac{\partial}{\partial r}(r^2\frac{\partial U}{\partial r}=0
\end{equation}
Durch zweilaige Integration nach $r$ erhalten wir schließlich:
\begin{equation}
    U=\frac{\alpha}{r}+\beta
\end{equation}
In diesem Fall sind $\alpha$ und $\beta$ Konstanten, die es noch zu ermitteln gilt. Setzt man nun vorraus, dass $U$ in unednlich großem Abstand Null sein soll, folgt sofort: $\beta=0$ und es bleibt:
\begin{equation}
    U=\frac{\alpha}{r}
\end{equation}
für die aüßere Lösung.
\newline
Nun zur inneren Lösung.
Gleichung 120 lässt sich umschreiben, zu:
\begin{equation}
    \frac{\partial}{\partial r}(r^2\frac{\partial U}{\partial r}=4\pi G\mu r^2
\end{equation}
Und erneut, durch zweimalige Integration, erhalten wir:
\begin{equation}
    U=\frac{2}{3}\pi G\mu r^2-\frac{A}{r}+B
\end{equation}
A und B sind wieder zu ermittlende Konstanten. Wir setzten nun wiederum vorraus, dass $U$ bei $r=0$ regulär sein soll, woraus direkt folgt $A=0$ und $U(0)=U_0=B$. Und somit:
\begin{equation}
    U=\frac{2}{3}\pi G\mu r^2+U_0
\end{equation}
Wir wollen nun die Konstanten bestimmen. Dafür betrachten wir die beiden Lösungen $U_A$ als die äußere und $U_I$ als die innere Lösung. Da der Übergang des Potentials, von der inneren zur äußeren Lösung glatt sein soll, ergibt sich an der Stelle $r=r_0$:
\begin{equation}
    \frac{dU_A}{dr}=\frac{dU_I}{dr}
\end{equation}
Also:
\begin{equation}
    -\frac{\alpha}{r_0^2}=\frac{4}{3}\pi G\mu r_0
\end{equation}
setzen wir nun $M=\frac{4}{3}\pi r_0^3\mu$, erhalten wir:
\begin{equation}
    \alpha=-GM\rightarrow U_A=-\frac{GM}{r}
\end{equation}
Für $U_I$ verwenden wir den Fakt, dass bei $r=r_0$ $U_A=U_I$ gelten muss:
\begin{equation}
    -\frac{GM}{r_0}=\frac{2}{3}\pi G\mu r_0^2+U_0\rightarrow U_0=-\frac{3GM}{2r_0}
\end{equation}
Damit ergbit sich das Potential im Inneren des Sterns zu:
\begin{equation}
    U_I=\frac{GM}{2r_0}(\frac{r^2}{r_0^2}-3)
\end{equation}
Die gravitative Bindungsenergie eines Sterns, berechnet sich mit der Formel:
\begin{equation}
    W_{pot}=\frac{4\pi}{2}\int_{0}^{r_0} U\mu r^2\, dr
\end{equation}
Dies bedeutet:
\begin{equation}
    W_{pot}=\frac{4\pi}{2}\frac{GM}{2r_0}\int_{0}^{r_0} \frac{r^2}{r_0^2}-3\, dr
\end{equation}
Und somit:
\begin{equation}
    W_{pot}=-\frac{8\pi}{3}GM
\end{equation}
\newline




>>>>>>> origin/master






\nocite{meinel}
\nocite{piotr}
\nocite{oloff}
\printbibliography
\end{document}
