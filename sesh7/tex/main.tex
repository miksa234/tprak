\documentclass[a4paper]{article}

\usepackage[T1]{fontenc}
\usepackage[utf8]{inputenc}

\usepackage{mathptmx}

\usepackage[a4paper, total={6in, 8in}]{geometry}
\usepackage{subcaption}
\usepackage[shortlabels]{enumitem}
\usepackage{amssymb}
\usepackage{amsthm}
\usepackage{mathtools}
\usepackage{braket}
\usepackage{bbm}
\usepackage{graphicx}
\usepackage{float}
\usepackage{yhmath}
\usepackage{tikz}
\usetikzlibrary{calc,decorations.markings}
\usepackage[colorlinks=true,naturalnames=true,plainpages=false,pdfpagelabels=true]{hyperref}
\usepackage[parfill]{parskip}
\usepackage[backend=biber, sorting=none]{biblatex}
\newcommand{\hbbar}{{\raisebox{0.05ex}{$\mathchar '26$}\mkern -9mu\raisebox{-0.15ex}{$\mathchar '26$}\mkern -9muh}}
\addbibresource{uni.bib}
\pagestyle{myheadings}
\markright{Popovic, Vogel\hfill Teleportation and Quantum Cryptography \hfill}


\title{University of Vienna\\ Faculty of Physics\\
\vspace{1.25cm}Lab-Course Theoretical Physics 2021S \\ Dispersion relations
}
\author{Milutin Popovic \& Tim Vogel \vspace{1cm}\\ Supervisor: Beatrix Hiesmyr}
\date{11. Juli, 2021}

\begin{document}
\maketitle
\noindent\rule[0.5ex]{\linewidth}{1pt}
\begin{abstract}
In this report we explore the ideas of teleportation and cryptography. First we discuss what teleportation means in a classical sense, we differ between transporting matter and transporting information. Then we prospect teleportation in the quantum world with a quantum computer. Finally we introduce the OTP encryption technique and explore the possibility of reducing the risk of an eves dropper by choosing the ansatz to generate a key with help of quantum theory, in cases of two and then three mutually unbiased bases.
\end{abstract}
\noindent\rule[0.5ex]{\linewidth}{1pt}

\tableofcontents
\section{Teleportation}
\subsection{Classical Teleportation}
In science fiction books, movies and video games the idea of teleportation,
that is almost instantaneous transportation, has been endorsed to full extent.
Real life realization of this concept would bring enormous advantages and solve
a lot of problems for humanity. This is partly allready put been into practice via
3D printers, but imagine we have constructed a technology that would allow us
to send matter over large distances almost instantaneously. We can discuss a
scenario where we would like to send an object or even a human body through
this apparatus, questions instantly arise.  At which level would we decompose
the object so , how would we decompose and/or recompose it, how long would it take to send
the pieces from one location to another etc. ? Let's say we want to send a
human, decomposing it on the organ level doesn't really come into question
since the transportation time would be the same as for the whole human.
Going lower than the atom level decomposition would just take up too much
energy, since we would have to split up the atom in electrons, protons and
neutrons. The atom level decomposition sounds most promising since the
molecules are just too massive and would also require a lot of energy to
accelerate and wouldn't make the assembly process easier.

Now, how fast can we send the human body on the atom level, how long would it
take. An average human body is about $70\ \text{kg}$ and contains
approximately $7\cdot 10^{27}$ \cite{body}.
Let's say we can accelerate these atoms to  $99\%$ the speed of light
($v=0.99c$). But to accelerate atoms we need to ionize them, so we would need
to send an extra electron with the atom.
To transport one such atom of the human body from one location to the other
we quickly notice that distance is not a priority factor, e.g from earth to sun
(distance $150\cdot10^9\ \text{m}$), it would take around $t = 8.5\ \text{min}$
for one such atom. But we need to send $7\cdot 10^{27}$ atoms, so sending them
one by one is not an option. If we can send one mole of particles ($N_A = 6.022
\cdot 10^{23}$)  per second this would reduce the time required to transport
them significantly to about three hours and 15 minutes plus the transportation
time of one such atom to the destination. However if we sent all the atoms of a
human body, we would still need to deploy the information needed to reconstruct
it. In that case we could think about only sending the blueprint, since atoms
are not unique and can be a prerequisite for teleportation
on the other side of the teleport.

The idea of the blueprint type approach would be to map the human body
($2x1x1m$) to a discrete space with a resolution of one lattice position
($10^{-10}$) which is either filled with an atom (hydrogen, oxygen, calcium,
kalium) or doesn't contain an atom at all.  This would allow us only to send
information, that is the blueprint, to the destination where we would
ultimately reconstruct the human body. This can be done by sending bits of
information, for instance one lattice position only requires three bits since
we need to encode 5 pieces of information, $\log_2(5) \simeq 3$.We can't round
down since we need to store at least 5 pieces of information . Alternatively we
could encode the information in two bits, or no bits at all. Meaning if the
encoder finds two bits then there is an atom of a specific kind and if it finds
no bits there is no atom. On the other hand a number with the resolution of
$10^{-10}$ has $10^{10}$ possibilities and can be encoded in $\log_2(10^{10})
\simeq 34$ bits. We have $2\cdot10^{30}$ positions to cover, ultimately needing
around $\log_2((2\cdot10^{30})^5) \simeq 504$ bits. A light pulse frequency of
$5\cdot 10^{14}$ this would take us about $12\ \text{ps}$ to send all those
bits.

\subsection{Quantum Teleportation}
Consider a scenario where Alice is given a unknown quantum system, e.g. a qubit.
Even though she is not interested in what state the system is, she knows that
Bob wants the same quantum system. By examining different scenarios she
concludes that sending the qubit through a quantum channel, there will be a non
vanishing possibility that the state is changed. Thus she wants to send Bob the
information needed to construct an accurate copy of the quantum system.

Let's say that both Alice and Bob own an qubit, where the overall quantum state
is an entangled state, e.g. one of the Bell states.
\begin{align}
    |e_1\rangle_{AB} := |\psi^-\rangle_{AB} =
    \frac{1}{\sqrt{2}}\bigg(|0\rangle_A \otimes|1\rangle_B - |1\rangle_A
    \otimes |0\rangle_B\bigg),\\
    |e_2\rangle_{AB} := |\psi^+\rangle_{AB} =
    \frac{1}{\sqrt{2}}\bigg(|0\rangle_A \otimes|1\rangle_B + |1\rangle_A
    \otimes |0\rangle_B\bigg),\\
    |e_3\rangle_{AB} := |\phi^-\rangle_{AB} =
    \frac{1}{\sqrt{2}}\bigg(|0\rangle_A \otimes|0\rangle_B - |1\rangle_A
    \otimes |1\rangle_B\bigg),\\
    |e_4\rangle_{AB} := |\phi^-\rangle_{AB} =
    \frac{1}{\sqrt{2}}\bigg(|0\rangle_A \otimes|0\rangle_B + |1\rangle_A
    \otimes |1\rangle_B\bigg),\\
\end{align}
Without loss of generality we choose the state $|\phi^+\rangle_{AB}$ that Alice
and Bob have and the unknown quantum state
\begin{align}
    |\phi\rangle_G = \alpha |0\rangle + \beta |1\rangle
\end{align}
with the normalization condition $|\alpha|^2 + |\beta|^2 = 1$.

We can describe these three particles with the following state
\begin{align}
    |\Phi\rangle_G \otimes |\phi^+\beta_{AB} &= (\alpha|0\rangle_G +\beta |1\beta_G)
    \otimes \frac{1}{\sqrt{2}} (|00\rangle_{AB} + |11\rangle_{AB})=\\
    &= \frac{1}{\sqrt{2}} (\alpha |00\rangle_{GA}\otimes |0\rangle_B
    \alpha|01\rangle_{GA}\otimes |1\rangle_B\\
    &\;\;\;\;\;\;\;\;\;\;\;\;+\beta |10\rangle_{GA}\otimes |0\rangle_B + \beta
    |11\rangle_{GA}\otimes
    |1\rangle_B)\label{eq:factor}.
\end{align}
To formally write $\alpha$ and $\beta$ at Bobs states we need to
express the equation as the tensor product of the kets with the subscript
$GA$ with the subscript $B$, which can be done by rewriting the $GA$ kets as
a superposition of the Bell states
\begin{align}
    |00\rangle_{GA} &= \frac{1}{\sqrt{2}}(|\phi^+\rangle_{GA}+
    |\phi^-\rangle_{GA}),\\
    |11\rangle_{GA} &= \frac{1}{\sqrt{2}}(|\phi^+\rangle_{GA} -
    |\phi^-\rangle_{GA}),\\
    |01\rangle_{GA} &= \frac{1}{\sqrt{2}}(|\psi^+\rangle_{GA}+
    |\psi^-\rangle_{GA}),\\
    |10\rangle_{GA} &= \frac{1}{\sqrt{2}}(|\psi^+\rangle_{GA}-
    |\psi^-\rangle_{GA}).
\end{align}

Substituting these into equation \ref{eq:factor} we get
\begin{align}
    |\Phi\rangle_G \otimes |\phi^+\rangle_{AB} = \frac{1}{2} \big(&
     |\psi^-\rangle \otimes ( -\beta|0\rangle + \alpha |1\rangle)\\
    +&|\psi^+\rangle \otimes ( \beta|0\rangle +  \alpha|1\rangle)\\
    +&|\phi^-\rangle \otimes ( \alpha|0\rangle -\beta |1\rangle)\\
    +&|\phi^+\rangle \otimes ( \alpha|0\rangle +  \beta|1\rangle)
        \big).
\end{align}
Thus the unitäry transformation operators are
\begin{align}
    U_{1} =
    \begin{pmatrix}
        0 & -1\\
        1 & 0
    \end{pmatrix} \;\;\;\;\;\;\;
    U_{2} =
    \begin{pmatrix}
        0 & 1\\
        1 & 0
    \end{pmatrix}\\
    U_{3} =
    \begin{pmatrix}
        1 & 0\\
        0 & -1
    \end{pmatrix} \;\;\;\;\;\;\;
    U_{4} =
    \begin{pmatrix}
        1 & 0\\
        0 & 1
    \end{pmatrix}.
\end{align}
With this Alice can send the unknown state $|\phi\rangle_G$ to Bob. For this
Alice needs to make a measurement in one of the Bell basis components
$\{|e_\lambda\rangle\}_{\lambda=1,2,3,4}$. Depending on this she operates on
the ket via $U_\lambda |\phi\rangle_B$ and bob needs to preform the opposite
measurement of $U_\lambda^\dagger$.

What if we would like to find out what state Bob possesses before or after Alice
performing the measurement on the system, without Bob knowing the outcome of
the measurement.

Before the measurement Bob's state is
\begin{align}
    \varrho_{GAB} &=
    (|\Phi\rangle_G \otimes |\phi^+\beta_{AB})(\langle\Phi|_G^\dagger\otimes
    \langle\phi^+|_{AB}^\dagger)=\\
    &=
\begin{pmatrix}0 & 0 & 0 & 0 & 0 & 0 & 0 & 0\\0 & \frac{|\beta|^{2}}{2} & -
    \frac{|\beta|^{2}}{2} & 0 & 0 & \frac{\alpha\beta^*}{2} & -
    \frac{\alpha\beta^*}{2} & 0\\0 & -
\frac{|\beta|^{2}}{2} & \frac{|\beta|^{2}}{2} & 0 & 0 & - \frac{\alpha
    \beta^*}{2} & \frac{\alpha \beta^*}{2} &
0\\0 & 0 & 0 & 0 & 0 & 0 & 0 & 0\\0 & 0 & 0 & 0 & 0 & 0 & 0 & 0\\0 &
    \frac{\alpha^*
\beta}{2} & - \frac{\alpha^* \beta}{2} & 0 & 0 & \frac{|\alpha|^{2}}{2} & -
    \frac{|\alpha|^{2}}{2} & 0\\0 &
- \frac{\alpha^* \beta}{2} & \frac{\alpha^* \beta}{2} & 0 & 0 & -
    \frac{|\alpha|^{2}}{2} & \frac{|\alpha|^{2}}{2} &
0\\0 & 0 & 0 & 0 & 0 & 0 & 0 & 0\end{pmatrix}.
\end{align}
taking the trace over the $GA$ components gets us
\begin{align}
    \text{Tr}_{GA}(\varrho_{GAB}) = \frac{1}{2}
    \begin{pmatrix}
        1 & 0 \\
        0 & 1
    \end{pmatrix}
\end{align}
where we used $|\alpha|^2 + |\beta|^2 = 1$.

On the other hand after the measurement, Bob still doesn't know the outcome
and cannot apply the measurement $U_\lambda^\dagger$ for a
$\lambda$ provided by Alice. With this Bob has a chance of $\frac{1}{4}$th to
get the state $|\phi\rangle_G$.
\subsection{Teleportation on a Quantum computer}
We want to consider an algorithm, consisting of Hadamard gates, CNOT gates, and a measurement in the computational basis $(\ket{0},\ket{1})$ at the end. The Hadamard gates are given as $H=\frac{1}{\sqrt{2}}\begin{pmatrix}
    1 & 1 \\
    1 & -1 \\
\end{pmatrix}$
and the CNOT gates as $\ket{0}\bra{0}\bigotimes\mathbb{I}+\ket{1}\bra{1}\bigotimes\sigma_1$.
The algorithm contains the following steps:
We input a set of three qubits, all in the state $\ket{0}$. We shall name them $q_1,q_2$ and $q_3$. At first we apply a Hadamard gate to $q_3$, followed by a CNOT gate to $q_2$ and $q_3$ in the second step. Then we apply a CNOT gate to the first two qubits, whereas a Hadamard gate is applied to the third one. For the last three steps, we start by applying a Hadamard gate onto the first two qubits in step three. For the fourth step we use a CNOT gate onto $q_2$ and $q_3$ and finally, we apply a Hadamard gate to the same two qubits and end with a measurement. So, if we input the state $I=\ket{000}$, going through all the above mentioned steps, we obtain as a result:
\begin{equation}
    O=\frac{1}{2}(\ket{000}+\ket{010}+\ket{100}+\ket{110})
\end{equation}
This means, Bob obtains the original $\ket{0}$ in any way he measures, and thus a teleportation is successful. What if we now consider an arbitrary state, given by:
\begin{equation}
    \ket{\psi}=\alpha\ket{0}+\beta\ket{1}=q_1
\end{equation}
This means, our input now looks like
\begin{equation}
    (\alpha\ket{0}+\beta\ket{1})\ket{00}
\end{equation}
As to expect, this input yields a different result:
\begin{equation}
    O=\frac{\alpha}{2}(\ket{000}+\ket{010}+\ket{100}+\ket{110)}+\frac{\beta}{2}(\ket{001}+\ket{011}-\ket{101}-\ket{111})
\end{equation}
\begin{equation}
    =\frac{1}{\sqrt{2}}((\ket{\phi^+}+\ket{\psi^+})\alpha\ket{0}+(\ket{\phi^-}+\ket{\psi^-})\beta\ket{1})
\end{equation}
This again completes a successful teleportation, since Bob will measure the Eigenvalue of $\ket{0}$ with a probability of $|\alpha|^2$ and the one of $\ket{1}$ with a probability of $|\beta|^2$, which is the same as measuring the initial state in the first place.

\section{Quantum Cryptography}
\subsection{One-Time-Pad}
Nowadays computers use sequences of bits, of 0's and 1's to encode information.
To represent the English alphabet, which consists of 26 letters, in a sequence
of bits, we need a series of 0's and 1's that have 26 possible outcomes. With
$N$ bits we can represent $2^N$ possible outcomes, which means the English
alphabet can be encoded in $\log_2(26) \rightarrow 5$ bits.  Even thought $2^5
= 32$, we need to keep in mind that we need to encode at least $26$ different
outcomes which would then leave $6$ outcomes empty.  Introducing capital
letters and special characters like ``!?\%\$\#...'', we ultimately need $8$
bits defining $1$ byte. This type of character encoding ($8$-bit) is the
standard for electronic communication and is referred as ``American Standard
Code for Information Interchange'', ASCII for short.

The One-Time-Pad (OTP) is a encryption technique, that uses a one time single
use pre-shared key. Let's say we want to encrypt the massage "BYE" in ASII
encoding this would be
\begin{align}
        010000100101100101000101.
\end{align}
With this bit sequence we add another randomly generated one which is called
the key. The operations used are the basic boolean algebra operations.
\begin{table}[H]
    \centering
    \begin{tabular}{l|c}
        Massage: &  $010000100101100101000101$\\
        Key:     &  $110101100110010101110011$ \\
        \hline
        Code:   &   $100101000011110000110110$
    \end{tabular}
\end{table}
The same key decrypts the code into the massage.
\begin{table}[H]
    \centering
    \begin{tabular}{l|c}
        Code : &  $100101000011110000110110$\\
        Key:     & $110101100110010101110011$ \\
        \hline
        Massage :   & $010000110100111101000011$
    \end{tabular}
\end{table}

It is mathematically proved, that if the key is really randomly generated, at
least the same length as the massage, only used once and only the sender and
the receiver are in possession of the key this encryption method is not
crackable.

\subsection{BB-84 Protocol}
To make OTP encryption more secure, a protocol based on quantum mechanics, the
BB-84 protocol is introduced. The BB-84 protocol allows us to generate a random key such that
it is much harder for an eavesdropper ``Eve'' to acquire it. The protocol calls for two
mutually unbiased bases and thus four possible outcomes.
This can be for example a polarized photon that is either
horizontally ($|H\rangle$), vertically ($|V\rangle$), $+45^\circ$ or
$-45^\circ$ polarized. We can translate $|H\rangle, |+45^\circ\rangle \equiv 0$ and
$|V\rangle, |-45^\circ\rangle \equiv 1$.  In the Bra-Ket notation we can write
\begin{align}
        |+45^\circ\rangle = \frac{1}{\sqrt{2}}(|H\rangle + |V\rangle),\\
        |-45^\circ\rangle = \frac{1}{\sqrt{2}}(|H\rangle - |V\rangle)
\end{align}
which can be achieved by rotating the polarization filter by $\pm 45^\circ$ in
the x-z axis.

All in all Alice has four states in her repertoire to choose from which
she then sends to Bob. Bob on the other hand can analyze the incoming photons
based on their polarization with a dual channel analyzer, where the $|H/V\rangle$
analyzer distracts photons left/right and the $|\pm45^\circ\rangle$ analyzer distracts
photons left/right.

To construct a randomly generated key Alice writes down
a random sequence of bits, for a each $0$ she chooses from the states $|H\rangle$ or
$|+45^\circ\rangle$ and for each $1$ she chooses from the states $|V\rangle$
and $|-45^\circ\rangle$. These polarized photons are sent to Bob's analyzer,
which leads to two possibilities, he either chose the correct basis on his
analyzer or he doesn't. Meaning that in the mean Bob chooses the right basis
$\frac{1}{2}$ of the time. After all photons are sent, Alice and Bob need to compare
the basis they chose (which they can do completely publicly) and cross of the
results where they choose the wrong basis. With this we have a randomly
generated key that can be used for OTP encryption.

The only way a third person, we called her Eve, can interfere is if she places
herself between Alice and Bob. For this Eve needs to measure the polarization
of the photon Alice sent, and then with this information send a photon to Bob.
If she doesn't send anything to Bob, Bob will notice somebody is interfering
and Eve would get caught. When the basis is then compared Eve will have the
same key Alice has.

Let us explore what would Bob get. Eve has a $\frac{1}{2}$ chance to choose the
same basis that Alice choose. This ultimately leads to a $\frac{1}{4}$ chance that Bob
and Alice have the wrong key, since Bob also needs to choose a basis.

To reduce the possibility that a third person is eavesdropping, Alice and Bob
need to compare a small part of their measurement results. This results in the
protocol needing to being carried out much longer for the sake of security.

\subsection{Six-State-Protocol}
The ``Six-state-protocol'' SSP is a generalization of the BB-84 protocol, where
instead of two, three mutually unbiased basis are used resulting in 6 states.
Thus, Alice and Bob have the chance of $\frac{1}{3}$ to choose the same basis,
meaning they would have to cross out $\frac{2}{3}$ of the results before
getting the key.

For Eve to not get detected there are two scenarios. In the first scenario she
would need to choose the same basis as Alice and Bob, that is a probability
of $\frac{1}{3}$ for each measurement. In the second scenario Eve
chooses the wrong basis but Bob measures in the right basis. Eve picking the
wrong basis happens with the probability of $\frac{2}{3}$, \textbf{after} that
Bob measuring in the correct basis would happen with the probability of
$\frac{1}{2}$, since we are dealing with MUBs.  Meaning Eve remains undetected
with a probability of $\frac{1}{3}$.

For Eve to get detected she would need to measure the wrong results
($\frac{2}{3}$) and \textbf{after} Bob needs to measure the wrong results
($\frac{1}{2}$). Ultimately Eve gets detected with the probability of
$\frac{1}{3}$.

The probability to detect Eve trivially scales with the number of measurements,
meaning $P(\text{detected}) = 1 - (\frac{2}{3})^n$ where $n$ are the
number of measurements. On the other hand the chance of not being detected
after $n$ measurements is $P(\text{not detected}) = 1 - P(\text{detected}$,
of course this doesn't mean Eve managed to get the right key. For Eve to pick
the get the right key she would also additionally need to choose the right
basis.

To even further minimize the risk of an eavesdropper Alice and Bob could
compare small parts of their qubit string like in the BB-84 protocol.
\nocite{six-state}
\nocite{teleportation}
\printbibliography
\end{document}
